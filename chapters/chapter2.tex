%versao de 15-MAR-2019

\chapter{Predicação}

Neste capítulo, discutiremos como um predicado se combina semanticamente com seu(s)
argumento(s). Olharemos primeiramente para sentenças com
verbos intransitivos, transitivos e bitransitivos. Em seguida,
passaremos a sentenças em que substantivos, adjetivos e
preposições assumem a função de predicados principais.
Quanto aos argumentos, limitaremos nossa atenção neste início
aos nomes próprios.


\section{Funções e a notação lambda}

O objetivo desta seção inicial é familiarizar o leitor com o conceito matemático de \textsc{função} e com a chamada \textsc{notação lambda}, que utilizaremos intensamente não apenas neste capítulo, mas em todo o livro. Fixaremos alguns pontos notacionais e terminológicos que serão cruciais na compreensão deste e dos demais capítulos.\\

\n \textbf{Funções} \\

\n Uma função pode ser vista como um dispositivo que efetua uma
transformação sobre uma determinada classe de objetos. De acordo
com essa caracterização mecânica ou dinâmica, uma função, ao
receber determinado objeto (chamado de \textsc{argumento} ou
\textsc{input}), retorna um outro objeto (chamado de
\textsc{valor} ou \textsc{output}). Para especificar uma função,
precisamos informar o tipo de objetos que ela aceita como
argumento, o tipo de objetos que ela retorna como valores e a
descrição da transformação efetuada. Chamamos de \textsc{domínio} da função o conjunto dos objetos que servem como argumento para a função. E de \textsc{contradomínio} da função o conjunto que especifica o
tipo de objeto que ela retorna como valor. Vejamos um exemplo. Seja \textit{f} uma
função que toma um número natural \textit{n} qualquer e retorna
o seu sucessor \textit{n + 1}. Nesse caso, tanto o domínio quanto
o contradomínio de \textit{f} são o conjunto dos números naturais
$\mathbb{N}$. Mas note que, como o número natural zero não é o
sucessor de nenhum número natural, não há nenhum argumento que o
retornará como valor. Chamamos de \textsc{imagem} de uma função o subconjunto do contradomínio que contém
apenas os objetos que são efetivamente retornados como valor de
algum argumento. No caso
de \textit{f}, a imagem é o conjunto dos números naturais
positivos. É comum representar $f$ da seguinte maneira:\\

\n \begin{tabular}{l l} 
	$f:$ & $\mathbb{N} \rightarrow \mathbb{N}$\\  
	     &  $x \mapsto x + 1$\\ 
   \end{tabular} \\

%\begin{align*}
	%f: &\ \mathbb{N} \rightarrow \mathbb{N} \\
	%&\ x \mapsto x + 1 
%\end{align*}

Já na literatura sobre semântica, utiliza-se para a representação de funções uma notação criada pelo
lógico americano Alonzo Church e conhecida como \textsc{notação
lambda} em razão da utilização da letra grega de mesmo nome
$\lambda$. Na notação lambda, nossa função \textit{f} do
parágrafo anterior seria
representada da seguinte forma:\\

\n $(\lambda n: n \in \mathbb{N}.\ n + 1)$\\

\n De maneira mais geral, estaremos lidando com expressões da
seguinte forma:\\

\n $(\lambda v:\phi .\ \gamma)$\\

\n Essa expressão representa a função que tem por domínio o
conjunto dos elementos $v$ que satisfazem $\phi$ e que transforma
cada um desses elementos em $\gamma$. Nessa expressão, $\lambda$ é
chamado de \textsc{operador lambda}, \textit{v} é a \textsc{variável}
ligada pelo operador lambda e $\gamma$ é chamado de \textsc{corpo}
da expressão ou da função.

É prática comum simplificar a notação acima, omitindo-se a
informação sobre o domínio da função, sempre que isso puder
ser inferido do contexto, bem como os parênteses, quando isso não resultar em ambiguidade. Por exemplo, se adotarmos 
a convenção de usar a letra \textit{n} para nos referirmos a
números naturais, então a representação acima pode ser simplificada:\\

\n $\lambda n.\ n + 1$\\

\n Uma expressão do tipo $(\lambda v. \phi)$ é chamada de
\textsc{abstração}-$\lambda$ ou \textsc{abstração funcional}.
Diz-se que tal expressão foi formada a partir da expressão $\phi$,
abstraindo-se sobre a variável \textit{v}. Nos casos que nos
interessam aqui, a expressão $\phi$ contém pelo menos uma
ocorrência da variável \textit{v}. No nosso exemplo
anterior, criamos a representação da função sucessor a partir
da expressão $n + 1$, abstraindo sobre a variável \textit{n}.

Para representarmos uma função $(\lambda v. \phi)$
aplicada a um argumento \textit{a}
qualquer de seu domínio, utilizamos uma expressão como a abaixo:\\

\n $(\lambda v.\ \phi)(a)$\\

\n Uma expressão desse tipo é chamada de \textsc{aplicação
funcional} ou simplesmente \textsc{aplicação}. Retornando ao
nosso exemplo, a aplicação da função sucessor ao número
natural 4 é
representada da seguinte forma:\\

\n $(\lambda n.\ n + 1)(4)$\\

\n A interpretação que desejamos para essa expressão é o
sucessor do número 4, ou seja, o número 5. Para obtermos esse
resultado, basta eliminarmos da representação da função o
prefixo $\lambda n.$ e substituirmos a variável \textit{n} pelo
número
4 no corpo da função:\\

\n $(\lambda n.\ n + 1)(4) \Rightarrow 4 + 1 \Rightarrow 5$\\

\n Esse processo de simplificação, que corresponde
intuitivamente à computação do valor de uma função aplicada a
um argumento, é chamado de \textsc{redução}-$\beta$. Mas é comum
também encontrarmos o termo \textsc{conversão}-$\lambda$, que é o
que adotaremos aqui.

Note ainda que a
escolha da variável \textit{n} nos exemplos acima foi, em um certo
sentido, arbitrária. Poderíamos igualmente ter utilizado \textit{m}
ou \textit{k}, por exemplo, que estaríamos ainda diante da mesma
função:\\

\n $(\lambda n.\ n + 1) \Leftrightarrow (\lambda m.\ m + 1)
\Leftrightarrow (\lambda k.\ k + 1)$\\

\n Esse processo de substituição da variável que segue o
operador lambda e de todas as suas ocorrências no corpo da
função por uma outra variável é chamado de
\textsc{redução}-$\alpha$ ou \textsc{conversão}-$\alpha$. Tanto a conversão-$\alpha$ quanto a conversão-$\lambda$ precisam estar sujeitas a determinadas condições para que produzam sempre os resultados desejados. Para saber mais sobre essas condições, ver o apêndice ao final do capítulo.\\

\n \textbf{Conjuntos e funções características}\\

\n Há um tipo especial de função que aparecerá com bastante
frequência neste livro e para cuja representação a notação
lambda também se mostra bastante adequada. Trata-se de funções
que mapeiam cada um dos elementos de seu domínio ou no valor 0 ou
no valor 1. Em outras palavras, o contradomínio dessas
funções é o conjunto $\{0,1\}$. Uma função desse tipo é
chamada de \textsc{função característica}. 

Para entender a razão
desse nome, vamos trabalhar em cima de um exemplo concreto. Seja
\textit{f} uma função de $\mathbb{N}$ em $\{0,1\}$ tal que
$f(n)=0$ se $n$ for ímpar e $f(n)=1$ se $n$ for par. Na notação
lambda, essa função pode ser representada da seguinte forma:\\


\n $\lambda n.\ n\ \text{é par}$\\


\n A convenção é a seguinte: sempre que o corpo da função for uma sentença (afirmação), estamos diante de uma função característica. Tal sentença funciona como condição necessária e suficiente para um argumento ser mapeado no valor 1. Assim, temos que\\

\n $(\lambda n.\ n\ \text{é par})(3) = 0$ já que a sentença \textit{3 é par} é falsa.\\

\n $(\lambda n.\ n\ \text{é par})(8) = 1$ já que a sentença \textit{8 é par} é verdadeira.\\

\n De maneira geral, temos que:\\

\n $(\lambda n.\ n\ \text{é par})(m) = 1$ \textit{sse} m for par.\\
 
\n Diz-se nesse caso que tal função caracteriza o conjunto dos números pares.

É importante ter em mente que lidar com conjuntos de elementos de um certo domínio ou com funções que levam elementos desse domínio em valores de
verdade (funções características) é, essencialmente, a
mesma coisa. Isso porque existe uma correspondência biunívoca (um a um) entre esses conjuntos e essas funções. Para cada função $F$ com domínio $D$, pode-se definir o conjunto $C$ (um subconjunto de $D$),  de modo que, para todo $x$ em $D$, tenhamos o seguinte:\\

\n $F(x) = 1\ sse\ x \in C $\\

\n Note, agora, que se dois conjuntos quaisquer $C_1$ e $C_2$ são distintos, então as funções correspondentes também serão distintas, já que haverá pelo menos um elemento que pertencerá a apenas um dos conjuntos, e que, portanto, será mapeado no valor 1 por apenas uma das funções. Da mesma forma, a duas funções características distintas corresponderão
sempre dois conjuntos distintos, já que pelo menos um elemento será mapeado por apenas uma delas no valor 1, e, portanto, pertencerá a apenas um dos conjuntos em questão. Em outras palavras, todo conjunto estará associado a uma única função característica, e toda função característica estará associada a apenas um único conjunto.

Dada essa correspondência biunívoca, podemos falar indistintamente em conjuntos ou em suas funções características sem perda de
informação, escolhendo o que nos for mais conveniente. Esteja,
portanto, preparado para conviver simultaneamente (às vezes em um
mesmo texto) com jargões baseados em conjuntos e jargões baseados em funções.\\

\n \textbf{Funções, conjuntos e predicados (propriedades) }\\

\n Além dessa relação estreita entre funções características e subconjuntos de um domínio, tiraremos partido de uma outra conexão. Tome, por exemplo, o conjunto dos números pares e a função $F$ que o caracteriza ($\lambda x.\ x\ \text{é par}$). Dizemos que os elementos desse conjunto têm a propriedade de ser par, ou que o predicado \textsc{par} se aplica a esses elementos. A aplicação de um predicado $P$ a um elemento $x$ qualquer é representada pela expressão $P(x)$.

Temos então uma conexão tripla entre funções, conjuntos e predicados. Para deixar clara essa conexão nas representações que usaremos neste livro, partindo de uma função característica $F$ qualquer, vamos representar seu conjunto característico por $F_C$ e o predicado correspondente por $F'$. Dessa forma, dado um elemento $x$ e uma função $F$ quaisquer, teremos sempre a equivalência abaixo:\\

\n $F(x)=1 \leftrightarrow x \in F_C \leftrightarrow F'(x)$\\

\n Tiraremos partido dessa dupla equivalência no decorrer de todo o livro, escolhendo a forma que nos for mais conveniente no momento. Em particular, a notação de predicados, por ser mais compacta, será usada com bastante frequência. Por fim, essa dupla equivalência também pode ser vista na sua forma negativa:\\

\n $F(x)=0 \leftrightarrow x \notin F_C \leftrightarrow \neg F'(x)$\\

\n O símbolo $\neg$ (negação) à frente do predicado é oriundo da lógica. $\neg F'(x)$ significa que $x$ não tem a propriedade $F'$ ou que o predicado $F'$ não se aplica a $x$. 
 

Após esse breve tutorial, já estamos aptos a prosseguir na interpretação de estruturas que envolvem predicados e seus argumentos. Sempre que o leitor estiver inseguro quanto à parte formal da discussão, poderá retornar a esta seção até que esses conceitos, terminologia e notações estejam bem fixados.


\section{Verbos intransitivos e aplicação funcional}

No capítulo anterior, consideramos sentenças que têm um verbo intransitivo como predicado e um nome próprio como sujeito, como (\ref{nno}) abaixo:

\begin{exe}
	\ex João trabalha.\label{nno}
\end{exe}

\n Analisamos a extensão do nome próprio como sendo o indivíduo portador do nome e a extensão do verbo como o conjunto dos indivíduos que trabalham. A relação sujeito-predicado foi então analisada como pertinência: a sentença é verdadeira se, e somente se, a extensão do sujeito pertence à extensão do predicado. \\

\n \den{João trabalha}  = 1 \textit{sse} $\text{João} \in \{x | x\ \text{trabalha}\}$.\\

\n \den{João trabalha}  = 1 \textit{sse} João trabalha.\\

\noindent Com relação aos nomes próprios, manteremos o que já temos:\\

\noindent \den{João} = João\\

Já com relação ao verbo \textit{trabalhar}, efetuaremos uma mudança, que poderá parecer arbitrária de início, mas que nos permitirá um maior poder de generalização na formulação de nossas regras composicionais, conforme veremos ainda neste capitulo. Em vez de assumir que a extensão de um verbo intransitivo denota um conjunto, vamos assumir que a mesma é uma função, no sentido matemático do termo, como acabamos de ver na seção anterior. Mais precisamente, assumiremos que a extensão desses verbos é uma função característica, que leva indivíduos em valores de verdade, 0 ou 1. No caso do verbo \textit{trabalhar}, por exemplo, sua extensão leva um indivíduo \textit{x} no valor de verdade 1 se \textit{x} trabalha e
no valor de verdade 0 se \textit{x} não trabalha. Usando a notação lambda, temos:\\

\n\den{trabalha} = $\lambda x:x\ \text{é um indivíduo}.\ x\ \text{trabalha}$\\

\n Assumindo que variáveis como \textit{x}, \textit{y}, \textit{z} tenham sempre indivíduos como valores, podemos simplificar:\\

\n\den{trabalha} = $\lambda x.\ x\ \text{trabalha}$\\

\noindent Convém notar, antes de mais nada, que observações semelhantes às que fizemos a respeito da relação entre extensão e significado no capítulo anterior se aplicam a esse
caso também. O significado do verbo \textit{trabalhar} não é igual à sua
extensão. Afinal, saber o significado desse verbo não é o mesmo
que saber, para cada indivíduo \textit{x}, se \textit{x} trabalha ou
não. Ao contrário, saber o significado deste verbo é saber as
condições necessárias e suficientes para que se possa dizer de
uma pessoa se é verdadeiro ou não que ela trabalha. Note, ainda, que
a representação acima da extensão do verbo \textit{trabalhar} faz menção a
essas condições.

Olhemos agora para a representação sintática de nossa
sentença.\\

\Tree [ [ [ João ].N ].NP [ [ trabalha ].V ].VP ].S

\bigskip

\n Já sabemos a extensão dos itens lexicais. Vamos assumir
agora que um nó não ramificado herda a extensão do nó
imediatamente dominado por ele, como já fizemos no capítulo
anterior.\\

\begin{tcolorbox}[boxrule=0pt,sharp corners]

\n \textbf{Princípio dos nós não ramificados}\\
Se $\alpha$ é um nó não ramificado cujo único constituinte
imediato é $\beta$, então \den{$\alpha$} = \den{$\beta$}.

\end{tcolorbox}

\bigskip

\n Assim, temos:\\

\n \den{NP} = \den{N} = \den{João} = João \\

\n \den{VP} = \den{V} = \den{trabalha} = $\lambda x.\ x\ \text{trabalha}$\\

\noindent Resta-nos agora determinar como a extensão de S é obtida
a partir das extensões de NP e VP. A extensão de NP é um
indivíduo, enquanto a extensão de VP é uma função que leva
indivíduos em valores de verdade. Como a extensão de S deve ser um valor
de verdade, o que precisamos é de um princípio que
derive a extensão de S, aplicando a extensão de VP à extensão do NP. Isto nos daria o seguinte resultado:\\

\n \den{S} = $(\lambda x.\ x\ \text{trabalha})$(João)\\

\n \den{S} = 1 \textit{sse} João trabalha\\

\noindent Isso é exatamente o que queríamos derivar. Um sistema
como o esquematizado acima se estende diretamente a outros casos
de sentenças formadas por um verbo intransitivo e um nome
próprio na posição de sujeito. Por exemplo, a extensão do nome
\textit{Maria} é o indivíduo chamado Maria, a de \textit{Pedro}, o indivíduo
chamado Pedro e assim por diante. Chamemos o conjunto de todos os
indivíduos de D$_{e}$, o domínio dos indivíduos (ou
entidades). Podemos afirmar então que a extensão de todo nome
próprio pertence a D$_{e}$. Com relação aos verbos
intransitivos em geral, podemos estender o que acabamos de ver com
o verbo \textit{trabalhar} para outros verbos dessa classe. Por exemplo,
assumiremos que a extensão do verbo \textit{estudar} é uma função $f$ que
leva indivíduos a valores de verdade, tal que, para todo indivíduo
$x$, $f(x)$ é igual a 1 se, e somente se, $x$
estuda. Se chamarmos o conjunto dos valores de verdade --- $\{0,1\}$ --- de
D$_{t}$, temos que a extensão de todo verbo intransitivo é
uma função de D$_{e}$ em D$_{t}$.\\

\n \den{estuda} = $\lambda x_{e}.\ x\ \text{estuda}$\\

\n Note que, na representação acima, escrevemos $\lambda x_{e}$
em vez de $\lambda x: x\in D_{e}$. É comum representar a
informação sobre o domínio de uma função através de um
subscrito junto à variável associada ao operador lambda,
exatamente como acabamos de fazer. Veremos mais exemplos disso no decorrer do livro.

Podemos formular agora aquele que será o pilar de todo nosso
sistema interpretativo, o princípio de composição conhecido
como \textsc{aplicação funcional}.\\

\begin{tcolorbox}[boxrule=0pt,sharp corners]

\n\textbf{Aplicação funcional}\\
Seja $\alpha$ um nó ramificado, cujos constituintes imediatos
são $\beta$ e $\gamma$. Se \den{$\beta$} é uma função e
\den{$\gamma$} pertence ao domínio de \den{$\beta$}, então
\den{$\alpha$} = \den{$\beta$}(\den{$\gamma$}).\\

\end{tcolorbox}

\bigskip

No exemplo mais acima, obtivemos a extensão da sentença \textit{João
trabalha} a partir das extensões de NP e VP, valendo-nos de
aplicação funcional. Isso foi possível porque o domínio da
função correspondente à extensão de VP era D$_{e}$, e a extensão
do NP sujeito pertencia a esse domínio.

\section{Verbos transitivos}

Passemos, agora, a um exemplo com um verbo transitivo:

\begin{exe}
    \ex João ama Maria. \label{jam}
\end{exe}

\Tree [ [ [ João ].N ].NP$_{suj}$ [ [ ama ].V [ [ Maria ].N
].NP$_{obj}$ ].VP ].S

\bigskip

\noindent A novidade aqui é o verbo \textit{ama}, que se combina com o objeto direto \textit{Maria}, formando um VP, e só depois com o sujeito \textit{João}, formando a sentença. Queremos que nosso sistema interpretativo respeite a estrutura sintática da sentença e proceda composicionalmente, atribuindo uma extensão a cada nó dessa estrutura.

Vejamos como proceder. Sabemos o que queremos derivar:\\

\noindent \den{S} = 1 \textit{sse} João ama Maria.\\

\noindent Já sabemos também que a extensão do NP sujeito é o
indivíduo João. Pensemos agora na extensão do VP
\textit{ama Maria}. Baseado no que vimos no exemplo anterior, obteremos o
resultado desejado se assumirmos que a extensão de VP é uma
função que leva indivíduos em valores de verdade. Neste caso,
queremos uma função que leve um indivíduo \textit{x} no valor de
verdade 1
se, e somente se, \textit{x} ama Maria:\\

\n \den{VP} = $\lambda x.\ x\ \text{ama Maria}$\\

\n Passando para o interior de VP, já sabemos que a extensão do NP
objeto é o indivíduo Maria. Nossa meta agora é descobrir uma
extensão adequada para o verbo \textit{amar}. Aqui podemos nos valer mais
uma vez do princípio de aplicação funcional como guia. Se
quisermos aplicá-lo neste caso, devemos assumir que a extensão de
V, que é igual à extensão de \textit{ama}, é uma função cujo domínio é
D$_{e}$. Mas que tipo de objeto essa função deve retornar quando
aplicada ao indivíduo Maria? Isso nós já sabemos: ela deve
retornar a extensão de VP, ou seja, uma outra função:\\

\noindent \den{V}(\den{NP$_{obj}$}) = \den{VP}\\

\noindent Queremos, portanto, uma função que, aplicada a um indivíduo, retorne uma outra função. Não há nada de errado com isso. Funções podem tomar como argumentos e retornar como valores quaisquer tipos de objetos, inclusive outras funções. Funções assim são chamadas, por vezes, de \textsc{funções de ordem superior}. A notação lambda fornece uma maneira bastante compacta de representar essas funções. Para o verbo \textit{ama}, temos:\\

\noindent \den{ama} = $\lambda x_{e}.\lambda y_{e}.\ y\ \text{ama}\ x$\\

\noindent Em palavras, a extensão de \textit{ama} aplicada a um indivíduo
\textit{x} retorna uma outra função. Essa outra função leva um indivíduo \textit{y}
no valor de verdade 1 se e somente se \textit{y} ama \textit{x}, sendo esse \textit{x} o argumento que alimentou a função original. Note que o argumento ($x$) dessa função original --- a extensão de \textit{ama} --- será fornecido pela extensão do objeto direto da sentença, que, nesse caso, corresponde à pessoa amada. Já o argumento ($y$) da função resultante --- a extensão de VP --- será fornecido pela extensão do sujeito. Daí termos $y$ ama $x$, e não $x$ ama $y$ na representação da função.

Para pôr em prática essas ferramentas que acabamos de adquirir, vamos
calcular, passo a passo, a derivação das condições de
verdade da sentença \textit{Maria odeia João}, cuja estrutura é análoga
à de \textit{João ama Maria}, mas cujo significado obviamente não o é.
As entradas lexicais que necessitamos estão dadas a seguir:\\

\n \den{Maria} = Maria\\

\n \den{João} = João\\

\n \den{odeia} = $\lambda x_{e}.\lambda y_{e}.\ y\ \text{odeia}\ x$\\

\n Com as entradas lexicais acima mais o princípio dos nós não
ramificados, obtemos a extensão do nó V e dos nós correspondentes
aos NPs sujeito e objeto.\\

\n \den{V} = \den{odeia}\\

\n \den{[$_{\text{NP}}$ João]} = \den{[$_{\text{N}}$ João]} =
\den{João}\\

\n \den{[$_{\text{NP}}$ Maria]} = \den{[$_{\text{N}}$ Maria]} =
\den{Maria}\\

\n Já a aplicação funcional nos permite obter a extensão de VP a
partir das extensões de V e do NP objeto:\\

\n \den{VP} = \den{V}(\den{[$_{\text{NP}}$ João]})\\

\n \den{VP} = $(\lambda x_{e}.\lambda y_{e}.\ y\ \text{odeia}\ x)$(João)\\

\n \den{VP} = $\lambda y_{e}.\ y\ \text{odeia João}$ \hfill (Conversão-$\lambda$)\\

\n Para obter as condições de verdade da sentença, utilizamos
aplicação funcional mais uma vez:\\

\n \den{S} = \den{VP}(\den{[$_{\text{NP}}$ Maria})\\

\n \den{S} = $(\lambda y_{e}.\ y\ \text{odeia João})$(Maria)\\

\n \den{S} = 1 \textit{sse} Maria odeia João \\

\n Podemos apreciar aqui um ponto vantajoso na utilização de funções para representar as extensões de predicados verbais. Apenas com aplicação funcional, fomos capazes de derivar as condições de verdade tanto de sentenças com verbos intransitivos quanto de sentenças com verbos transitivos. Conforme veremos mais adiante, outros tipos de predicados verbais e não verbais também receberão tratamento adequado via funções e aplicação funcional.

Antes de prosseguir, vamos introduzir uma notação que usaremos no restante deste livro e que, mais cedo ou mais tarde, o leitor que se aventurar pela literatura em semântica formal encontrará. Retomemos a extensão dos verbos \textit{trabalhar} e \textit{amar}, a título de exemplo:\\

\n \den{trabalhar} = $\lambda x.\ x\ \text{trabalha}$ \\

\n \den{amar} = $\lambda x.\lambda y.\ y\ \text{ama}\ x$ \\

\n É bastante comum nos textos da área utilizar, na parte da metalinguagem correspondente às especificações das condições de verdade, uma notação oriunda da lógica de predicados. Nela, sentenças simples formadas por um predicado (\textit{P}) e seus argumentos ($a_{1}, a_{2}, ..., a_{n}$) são representadas por $P(a_{1}, a_{2}, ..., a_{n})$. Nessa notação, o predicado aparece em primeiro lugar, seguido dos argumentos entre parênteses e separados por vírgulas. Assim, em vez de \textit{x trabalha} ou \textit{y ama x}, temos \textit{trabalha(x)} ou \textit{ama(y,x)}. Para diferenciar os predicados da linguagem objeto (no caso, o português)  desses predicados da metalinguagem, é comum representar esses últimos por algum tipo especial de fonte tipográfica, tais como negrito, itálico, caixa alta ou versalete. Aqui, optaremos por essa última. Dessa forma, as extensões vistas acima para os verbos \textit{trabalhar} e \textit{amar} receberão as seguintes representações:\\

\n \den{trabalhar} = $\lambda x.\ \text{\textsc{trabalha}}(x)$ \\

\n \den{amar} = $\lambda x.\lambda y.\ \text{\textsc{ama}}(y,x)$ \\

Deve-se sempre ter em mente que essa é uma questão meramente notacional, adotada por conveniência e tradição na área. Do ponto de vista técnico, as funções acima são as mesmas funções com que trabalhamos anteriormente.

\section{Tipos e domínios semânticos}

Vimos acima que a extensão de um nome próprio pertence a $D_{e}$, o
domínio dos indivíduos, e que a extensão de uma sentença pertence
a $D_{t}$, o domínio dos valores de verdade. Vamos introduzir agora
uma maneira de representar os domínios a que pertencem as
extensões de verbos transitivos e intransitivos, além de outras
variedades de constituintes que iremos discutir mais adiante.

Dizemos que a extensão de um constituinte sintático é de
\textsc{tipo} \textit{e} se ela pertence a $D_{e}$, e que a extensão de um
constituinte é de tipo \textit{t} se ela pertence a $D_{t}$. Para extensões funcionais, dizemos que a extensão de um constituinte é de tipo
$\langle\sigma ,\tau\rangle$ se ela é uma função que leva
elementos de $D_{\sigma}$ em $D_{\tau}$. Nesse caso, $D_{\langle\sigma,\tau\rangle}$ é o conjunto de todas essas funções. A extensão
de um verbo intransitivo é, então, de tipo $\langle e,t \rangle$, já
que é uma função que leva indivíduos em valores de verdade, e
$D_{\langle e,t \rangle}$ representa o domínio das funções
de tipo $\langle e,t \rangle$. Já a extensão de um verbo
transitivo é de tipo $\langle e,\langle e,t \rangle\rangle$,
sendo $D_{\langle e,\langle e,t\rangle\rangle}$ o domínio
das funções deste tipo. Como vimos, as extensões de verbos transitivos são funções que levam elementos de tipo \textit{e} a funções de tipo $\langle e,t \rangle$.

Podemos definir recursivamente um número infinito de
tipos semânticos, a partir dos tipos básicos \textit{e} e
\textit{t}. Por definição recursiva, entenda-se uma definição que assume a existência de tipos básicos (no nosso caso, \textit{e} e \textit{t}) e define tipos complexos (no nosso caso, os tipos funcionais) a partir deles. Vejamos como:\\

\begin{tcolorbox}[boxrule=0pt,sharp corners]

\n \textbf{Tipos semânticos}

\begin{itemize}

\item[(i)] \textit{e} e \textit{t} são tipos semânticos;

\item[(ii)] Se $\sigma$ e $\tau$ são tipos semânticos, então
$\langle\sigma ,\tau \rangle$ é um tipo semântico;

\item[(iii)] Nada mais é um tipo semântico.

\end{itemize}

\end{tcolorbox}

\bigskip

\n Pela cláusula (i), temos que tanto $e$ quanto $t$ são tipos semânticos. Sendo assim, de acordo com a cláusula (ii), temos que $\langle e,t \rangle$ também é um tipo semântico. Valendo-nos novamente da cláusula (ii), temos que $\langle e,\langle e,t \rangle\rangle$ também é um tipo semântico, já que tanto $e$ quanto $\langle e,t \rangle$ o são. O mesmo vale para $\langle t,t \rangle$, $\langle\langle e,t \rangle,t\rangle$, e infinitos outros tipos funcionais, alguns dos quais aparecerão no decorrer deste livro. Notemos, por fim, que a cláusula (iii) afirma que apenas os objetos construídos a partir de (i) e (ii) são tipos semânticos.

Quanto aos domínios semânticos, eles são pareados aos tipos que acabamos de ver:\\

\begin{tcolorbox}[boxrule=0pt,sharp corners]

\noindent \textbf{Domínios semânticos}\\

\n $D_{e}$: conjunto dos indivíduos

\n $D_{t}$: conjunto dos valores de verdade (\{0,1\})

\n $D_{\langle\sigma ,\tau \rangle}$: conjunto das
funções de $D_{\sigma}$ em $D_{\tau}$.

\end{tcolorbox}

\bigskip

\n Utilizaremos daqui em diante a notação acima para representar
o tipo semântico (ou o domínio) a que pertence a extensão de
um sintagma. Todos os itens lexicais e demais constituintes
sintáticos que discutiremos no decorrer deste livro possuirão
tipos semânticos que podem ser obtidos a partir da definição
recursiva que acabou de ser apresentada.


\section{Verbos bitransitivos}

Resumindo o que vimos até aqui, podemos dizer que nosso sistema se
baseia em um léxico, no qual estão especificadas as extensões de
todos os itens lexicais, e dois princípios de composição,
que fornecem instruções sobre como obter a extensão de um
constituinte sintático a partir de seus subconstituintes
imediatos: o princípio de aplicação funcional e o
princípio dos nós não ramificados. Vamos ver agora como esses
princípios se aplicam na obtenção do significado de
sentença com verbos bitransitivos, para os quais assumiremos a representação abaixo (mas ver exercício II ao final do capítulo).\\

\Tree [ [ [ João ].N ].NP$_{su}$ [ [ [ apresentou ].V [ [ Pedro ].N
].NP$_{od}$ ].V$^{\prime}$ [ [ para ].P [ [ Maria ].N ].NP$_{oi}$ ].PP ].VP ].S

\bigskip

\n Há dois itens lexicais nessa estrutura que ainda não
discutimos: a preposição \textit{para} e o verbo \textit{apresentou}. Com
relação à preposição, assumiremos tratar-se de um item
semanticamente vácuo, inserido na estrutura por razões puramente
sintáticas (caso gramatical, talvez). O sistema interpretativo desprezará
esse item e tratará o nó PP como se fosse não ramificado. Fica
faltando, então, uma extensão para o verbo \textit{apresentou}. Vamos
proceder de cima para baixo. O resultado final que desejamos
é:\\

\noindent \den{S} = 1 \textit{sse} João apresentou Pedro para
Maria.\\

\n Na notação importada da lógica de predicados, que começaremos a adotar a partir daqui, temos:\\

\noindent \den{S} = 1 \textit{sse} $\predica{apresentou}{joão,pedro,maria}$\\


\noindent Além de representar o predicado em versalete, representaremos os argumentos em itálico e minúsculas, demarcando melhor a distinção entre um nome próprio (João) e o indivíduo correspondente (\textit{joão}). Dessa forma, com relação ao sujeito, a entrada lexical de \textit{João}
mais o princípio dos nós não ramificados nos fornecem o
seguinte:\\

\noindent \den{NP$_{su}$} = \den{N} = \den{João} = \textit{joão}\\

\noindent Para que aplicação funcional possa nos servir, vamos
assumir que VP seja uma função de tipo $\langle e,t \rangle$ que
leva um indivíduo \textit{x} no valor de verdade 1 se e somente se
\textit{x} apresentou Pedro para Maria.\\

\noindent \den{VP} = $\lambda x_{e}.\ \predica{apresentou}{x,pedro,maria}$\\

\noindent Quanto a PP, como este será tratado como um nó não
ramificado, temos o seguinte:\\

\noindent \den{PP} = \den{NP$_{oi}$} = \den{N} = \den{Maria} =
\textit{maria}\\

\noindent E quanto a V$^{\prime}$? Novamente, guiados por aplicação
funcional, podemos deduzir seu tipo semântico. Estamos
buscando uma função que tomará a extensão de PP (tipo $e$) como
argumento e retornará a extensão de VP (tipo
$\langle e,t \rangle$) como valor, ou seja, uma função de tipo
$\langle e, \langle e,t \rangle\rangle$.\\

\noindent \den{V$^{\prime}$} = $\lambda y_{e}.\ \lambda x_{e}.\ \predica{apresentou}{x,pedro,y}$\\

\noindent Com relação ao objeto direto, temos que: \\

\noindent \den{NP$_{od}$} = \den{N} = \den{Pedro} = \textit{pedro}\\

\noindent Por fim, chegamos ao verbo. Este deverá ser uma função
que toma a extensão do objeto direto (tipo $e$) como argumento e
retorna a extensão de V$^{\prime}$ (tipo
$\langle e,\langle e,t \rangle\rangle$) como valor, ou seja, a extensão do
verbo \textit{apresentou} será uma função de tipo
$\langle e,\langle e,\langle e,t \rangle\rangle\rangle$.\\

\noindent \den{apresentou} = $\lambda z_{e}.\ \lambda y_{e}.\
\lambda x_{e}.\ \predica{apresentou}{x,z,y}$\\

\noindent A extensão do verbo \textit{apresentou} pode ser vista como uma função que toma três
indivíduos \textit{z, y, x} --- um de cada vez e nessa ordem --- como argumentos e retorna o valor de
verdade 1 se, e somente se, \textit{x} apresentou \textit{z} pra
\textit{y}. Outros verbos bitransitivos como \textit{dar} e \textit{mostrar}, por
exemplo, podem ser tratados da mesma maneira.

É crucial notar que o primeiro argumento ($z$) na representação acima está vinculado, por assim dizer, com a posição de objeto direto, o segundo ($y$) com a posição de objeto indireto e o terceiro ($x$) com a posição de sujeito. Isso reflete a estrutura sintática que assumimos, de acordo com a qual o verbo forma, primeiramente, um constituinte com seu objeto direto (V$^{\prime}$), unindo-se posteriormente com o objeto indireto (VP) e, por fim, com o sujeito, resultando em uma sentença (S). Queremos, assim, chamar a atenção ao fato de que a escolha da letra ($x$, $y$, ou $z$) para representar uma posição argumental não é, em si mesma, relevante. O que realmente importa é essa vinculação com as devidas posições. Dessa forma, poderíamos igualmente representar a extensão do verbo \textit{apresentar} das seguintes formas:\\

\n \den{apresentou} = $\lambda x.\ \lambda y.\
\lambda z.\ z\ \text{apresentou}\ x\ \text{para}\ y$\\

\n \den{apresentou} = $\lambda x.\ \lambda y.\
\lambda z.\ \predica{apresentou}{z,x,y}$\\

\n Trata-se da mesma função anterior, já que aqui também temos o primeiro argumento ($x$) vinculado à posição de objeto direto, o segundo ($y$) à de objeto indireto, e o terceiro ($z$) à de sujeito (a esse respeito, ver o exercício I ao final do capítulo).

\section{Predicados não verbais}

Nosso próximo passo será a análise de sentenças contendo
predicados não verbais, como substantivos comuns, adjetivos e algumas
preposições, como em (\ref{pnv}) abaixo:
%xl
\begin{exe}
\ex\label{pnv}
\begin{xlist}
\ex Maria é pianista.\label{pnva}
\ex João é feliz.\label{pnvb}
\ex Pedro está em Campinas.\label{pnvc}
\end{xlist}
\end{exe}

\n Como não discutiremos temas ligados a tempo e aspecto neste
curso, passaremos por cima da distinção entre \textit{ser} e \textit{estar}, e
assumiremos, apenas por conveniência, que ambos são sinônimos, sem
maiores justificativas. Isso é obviamente uma simplificação, mas
uma discussão mais aprofundada nos afastaria dos temas mais
centrais deste livro. Iremos nos referir tanto a \textit{ser} quanto a
\textit{estar} através de um termo mais genérico: cópula.

Vamos assumir para sentenças como (\ref{pnva}) e (\ref{pnvb})
as estruturas abaixo:\\

$\vcenter{%
	\hbox{\Tree [ [ [ Maria ].N ].NP [ [ é ].V [ [ pianista ].N ].NP ].VP ].S }} $ \hspace{1in} $ \vcenter{%
	\hbox{\Tree [ [ [ João ].N ].NP [ [ é ].V [ [ feliz ].A ].AP ].VP ].S }} $


\bigskip

\n Substantivos comuns e adjetivos como os acima guardam certa
semelhança com verbos intransitivos. Assim como alguém que sabe o
significado do verbo \textit{fumar} sabe as condições para que se
possa dizer de um indivíduo qualquer se ele fuma ou não, uma pessoa que
sabe o significado do substantivo comum \textit{pianista} ou do adjetivo \textit{feliz} sabe as condições para que se possa dizer de uma
pessoa qualquer se ela é ou não pianista ou se ela é feliz ou não. Dada essa
semelhança, iremos propor para esses predicados extensões de tipo
$\langle e,t \rangle$:\\

\n \den{pianista} = $\lambda x.\ x\ \text{é pianista}$\\

\n \den{feliz} = $\lambda x.\ x\ \text{é feliz}$\\

\n Na notação de predicados, costuma-se omitir o verbo \textit{ser}:\\

\n \den{pianista} = $\lambda x.\ \predica{pianista}{x}$\\

\n \den{feliz} = $\lambda x.\ \predica{feliz}{x}$\\

\n Assumiremos que a cópula é um elemento semanticamente vácuo, e
que, portanto, a extensão de VP nos exemplos acima é a mesma de AP e NP (mas ver exercício V ao final do capítulo):\\

\n \den{VP} = \den{NP} = $\lambda x_{e}.\ \predica{pianista}{x}$\\

\n \den{VP} = \den{AP} = \den{feliz} = $\lambda x_{e}.\ \predica{feliz}{x}$\\

\n Com isso, podemos nos valer de aplicação funcional para obter
as condições de verdade destas sentenças a partir das
extensões dos NPs sujeito e dos respectivos VPs. Para
(\ref{pnva}), temos:\\

\n \den{S} = \den{VP}(\den{NP})\\

\n \den{S} = ($\lambda x_{e}.\ \predica{pianista}{x}$)(\textit{joão})\\

\n \den{S} = 1 \textit{sse} $\predica{pianista}{joão}$\\

\n Para (\ref{pnvb}), procedemos de maneira inteiramente análoga.

Passemos agora a (\ref{pnvc}), que envolve um sintagma
preposicional:\\

\Tree [ [ [ Pedro ].N ].NP [ [ está ].V [ [ em ].P [ [ Campinas
].N ].NP ].PP ].VP ].S

\bigskip

\n O paralelo neste caso é entre a preposição \textit{em} e verbos
transitivos. Intuitivamente, ambos relacionam dois indivíduos.
Assim como alguém que sabe o significado do verbo transitivo
\textit{amar} sabe as condições para que se possa dizer de dois
indivíduos \textit{x} e \textit{y} se \textit{x} ama \textit{y} ou não,
uma pessoa que saiba o significado da preposição locativa \textit{em}
sabe as condições para que se possa dizer de dois indivíduos
\textit{x} e \textit{y} se \textit{x} está ou não em \textit{y}. A
entrada lexical que proporemos para essa preposição faz jus a
esse paralelo, tratando a extensão de \textit{em} como uma função de
tipo $\langle e, \langle e,t\rangle\rangle$.\\

\n \den{em} = $\lambda x_{e}.\ \lambda y_{e}.\ \predica{em}{y,x}$\\

\n O cálculo das condições de verdade de (\ref{pnvc}) procede
de maneira análoga aos casos envolvendo verbos transitivos. Para a
obter a extensão de PP, utilizamos aplicação funcional:\\

\n \den{PP} = \den{P}(\den{NP})\\

\n \den{PP} = ($\lambda x.\ \lambda y.\ \predica{em}{y,x}$)(\textit{campinas})\\

\n \den{PP} = $\lambda y.\ \predica{em}{y,campinas}$\\

\n Como a cópula é um item semanticamente vácuo, temos que a
extensão de VP é igual à extensão de PP. Por fim, utilizamos
aplicação funcional mais uma vez para obtermos as condições
de verdade da sentença:\\

\n \den{S} = \den{VP}(\den{NP})\\

\n \den{S} = ($\lambda y.\ \predica{em}{y,campinas}$)(\textit{pedro})\\

\n \den{S} = 1 \textit{sse} $\predica{em}{pedro,campinas}$\\

\n Além de adjetivos e nomes comuns como \textit{feliz} e \textit{pianista}, que
poderíamos chamar de intransitivos, há também casos que se
assemelham a verbos transitivos, como, por exemplo, \textit{orgulhoso} e
\textit{irmão}:

%xl
\begin{exe}
\ex\label{ptr}
\begin{xlist}
\ex João é irmão de Maria.\label{ptra}
\ex João está orgulhoso de Maria.\label{ptrb}
\end{xlist}
\end{exe}

\Tree [.S [.NP [.N João ] ] [.VP [.V é
		] [.NP [.N irmão ]  [.PP [.P de ] [.NP [.N Maria ] ] ] ] ]
		] \ \ \ \Tree [.S [.NP [.N João ] ] [.VP [.V está
		] [.AP [.A orgulhoso ]  [.PP [.P de ] [.NP [.N Maria ] ] ] ] ]
		]

\bigskip


\n Para esses casos, vamos assumir que a preposição \textit{de} é
semanticamente vácua. Isso permitirá que a extensão do NP
complemento de P nas estruturas acima possa servir de argumento
para a extensão de \textit{irmão} e \textit{orgulhoso}, que trataremos como
sendo funções de tipo $\langle e, \langle e,t \rangle\rangle$,
a exemplo de outros predicados transitivos que já
discutimos.\\

\n \den{irmão} = $\lambda x.\ \lambda y.\ \predica{irmão}{y,x}$\\

\n \den{orgulhoso} = $\lambda x.\ \lambda y.\ \predica{orgulhoso}{y,x}$\\

\n Fica como exercício para o leitor o cálculo, passo a passo, das
condições de verdade de (\ref{ptra}) e (\ref{ptrb}).

\section{Relações, funções e \textit{currying}}

Antes de fechar o capítulo, mais uma seção de cunho técnico, mas importante para o leitor se familiarizar com certas práticas e jargões da área.

Notamos na seção inicial deste capítulo que há uma correspondência biunívoca entre conjuntos e funções características construídos a partir de certo domínio. Nos casos que analisamos até aqui, o domínio relevante é $D_e$, o domínio dos indivíduos. No caso dos verbos intransitivos, notamos correspondências como as abaixo, em que $a$ é um indivíduo qualquer:\\

\n \den{trabalha}($a$) = 1 $\leftrightarrow$ $a \in \{x\ |\ x\ \text{trabalha} \}$ \\

\n \den{trabalha}($a$) = 0 $\leftrightarrow$ $a \notin \{x\ | \ x\ \text{trabalha} \}$ \\

\n Tais equivalências se
estendem para os casos de verbos transitivos e bitransitivos. Entretanto, em vez de caracterizar conjuntos de indivíduos, as extensões dos verbos transitivos e bitransitivos caracterizam \textsc{relações} entre indivíduos. Relações nada mais são que conjuntos de sequências ordenadas, também chamadas de \textsc{tuplas}. Assim, verbos transitivos são frequentemente apresentados como denotando conjuntos de pares ordenados (relações binárias) e os bitransitivos como denotando conjuntos de ternos ordenados (relações ternárias), como ilustrado abaixo para os verbos \textit{amar} e \textit{apresentar}, respectivamente:\\

\n $R_{amar} = \{\langle x,y \rangle | x\ \text{ama}\ y\}$\\

\n $R_{apresentar} = \{\langle x,y,z \rangle | x\ \text{apresenta}\ y\ \text{para}\ z\}$\\

\n Os colchetes angulados ($\langle \rangle$) indicam tratar-se de uma sequência ordenada e que, portanto, $\langle x,y \rangle$ será diferente de $\langle y,x \rangle$ sempre que $x$ for diferente de $y$. No caso de \textit{amar}, por exemplo, um par $\langle a,b \rangle$ pertencerá à relação sempre que o primeiro membro do par ($a$) amar o segundo ($b$).

Novamente, essa mudança de funções para relações é inofensiva, se atentarmos para as seguintes equivalências, em que $a$, $b$ e $c$ são três indivíduos quaisquer: \\

\n \den{amar}($a$)($b$) = 1 $\leftrightarrow$ $\langle b,a \rangle \in R_{amar}$ \\

\n \den{apresentar}($a$)($b$)($c$) = 1 $\leftrightarrow$ $\langle c,a,b \rangle \in R_{apresentar}$ \\

Há ainda um ponto relacionado ao que acabamos de ver e que é importante o leitor conhecer. Primeiramente, notemos que, a partir da função correspondente à extensão de um verbo transitivo, para chegarmos a um valor de verdade, procedemos em dois passos. Inicialmente, aplicamos a função a um indivíduo, obtendo uma outra função. Essa outra função, por sua vez, quando aplicada também a um indivíduo, retorna um valor de verdade (0 ou 1). \\

\n \den{amar}($x$)($y$) = 1 \textit{sse} $y$ ama $x$ \\

\n O mesmo se deu com verbos bitransitivos, sendo que, nesse caso, o processo envolveu três passos. \\

\n \den{apresentar}($x$)($y$)($z$) = 1 \textit{sse} $z$ apresenta $x$ para $y$ \\

\n Não fosse nosso intuito de fazer a derivação semântica corresponder à estrutura sintática binária que estamos assumindo, poderíamos reduzir esses processos a um único passo, identificando as extensões verbais com funções que tomam sequências (tuplas) de indivíduos retornando diretamente um valor de verdade. Dessa forma, em vez de denotações como as em (a) nos exemplos abaixo, teríamos aquelas em (b):

\begin{exe}
	\ex
	\begin{xlist}
\ex \den{ama$_{1}$} = $\lambda x.\lambda y.\ y\ \text{ama}\ x$
\ex \den{ama$_{2}$} = $\lambda\langle x,y \rangle.\  y\ \text{ama}\ x$
	\end{xlist}
\end{exe}

\begin{exe}
	\ex
	\begin{xlist}
		\ex \den{apresenta$_{1}$} = $\lambda x.\lambda y.\lambda z.\ z\ \text{apresenta}\ x\ \text{pra}\ y$
		\ex \den{apresenta$_{2}$} = $\lambda\langle x,y,z \rangle.\ z\ \text{apresenta}\ x\ \text{pra}\ y$
	\end{xlist}
\end{exe}

\n No caso de \textit{amar}, o argumento da função seria um par ordenado, enquanto no caso de \textit{apresentar}, teríamos um terno ordenado. Matematicamente, essa mudança é inofensiva se atentarmos para as seguintes igualdades, válidas para quaisquer $x$, $y$, $z$: \\

\n \den{ama$_1$}($x$)($y$) = \den{ama$_2$}($\langle x,y \rangle$)\\

\n \den{apresenta$_1$}($x$)($y$)($z$) =  \den{apresenta$_2$}($\langle x,y,z \rangle$)\\ 

\n Essas igualdades se generalizam para funções com qualquer número de argumentos. A transformação de uma função \textit{f} qualquer que toma uma sequência $\langle a_{1}, a_{2},...,a_{n}\rangle$ como argumento em uma outra
função $f'$ que toma \textit{n} argumentos, $a_{1},
a_{2},...,a_{n}$, um de cada vez, obedecendo à equivalência acima é chamada de \textit{curryamento (currying)} ou \textit{shönfinkelização}, em homenagem a Haskell Curry e Moses Schönfinkel, pioneiros na descrição e aplicação desse processo. A importância de conhecer essas representações e essas equivalências é que a literatura da área está cheia de alternâncias entre elas e os seus respectivos jargões.


\section*{Apêndice: mais sobre conversões lambda e alfa}
\addcontentsline{toc}{section}{Apêndice: mais sobre conversões lambda e alfa}

Apresentamos no início deste capítulo a notação lambda para a representação de funções e os processos de conversão-$\lambda$ e conversão-$\alpha$. Vejamos agora alguns cuidados necessários na manipulação desses
processos. Comecemos pela conversão-$\lambda$, fazendo  algumas
considerações preliminares. Se alguém nos pergunta a que a
expressão $2+3$ se refere, não temos dúvida em responder que se
refere ao número 5. Se, no entanto, nos é perguntado a que a
expressão $2+n$ se refere, ficamos sem uma resposta direta. A
razão é óbvia: essa expressão contém uma ocorrência livre de
uma variável e, a menos que nos seja dito que valor atribuir a
essa variável, não é possível identificar a que a expressão se
refere.

Expressões mais complexas envolvendo funções e o operador
lambda estão sujeitas ao mesmo tipo de comentário. Considere,
por exemplo, as duas funções abaixo:

\begin{exe}
	\ex $\lambda n.\ n+1$ \label{ok}
\end{exe}

\begin{exe}
	\ex $\lambda n.\ n+m$ \label{ol}
\end{exe}

\n No caso de (\ref{ok}), está claro diante de que função nós
estamos: a função sucessor. Já no caso de (\ref{ol}), a menos
que nos seja informado que valor atribuir a variável \textit{m},
ficamos sem saber com que função estamos lidando. Novamente, a
razão para isto está no fato de que a variável \textit{m} aparece
livre na expressão em questão. Variáveis livres podem aparecer não
apenas no interior de uma abstração, mas também na expressão do
argumento aplicado a uma função. Compare os dois casos a seguir:

\begin{exe}
	\ex  $(\lambda n.\ n+1)(3)$ \label{bn}
\end{exe}

\begin{exe}
	\ex  $(\lambda n.\ n+1)(m)$ \label{cn}
\end{exe}

\n Em (\ref{bn}), temos a aplicação da função sucessor ao
número 3. Não há variáveis livres e a expressão deve ter um valor
definido. Ao simplificar essa expressão através da
conversão-$\lambda$ isso
fica claro:\\

\n $(\lambda n.\ n+1)(3)$\\
$3+1$\\
$4$\\

\n No caso de (\ref{cn}), a expressão contém uma variável livre e, portanto, a ela não podemos atribuir um valor definido. Novamente,
a conversão-$\lambda$ deixa isso claro:\\

\n $(\lambda n.\ n+1)(m)$\\
$m+1$\\

\n Note a coerência nos resultados obtidos. A expressão que não
continha variáveis livres antes da conversão-$\lambda$ (e que,
portanto, tinha um valor definido) foi convertida em uma outra
expressão também sem variáveis livres, continuando a ter um valor
definido. Já a expressão que continha uma variável livre antes da
conversão-$\lambda$ foi convertida em uma expressão que também
contém uma variável livre, continuando portanto sem um valor
definido. Esta é uma característica da conversão-$\lambda$:
expressões são simplificadas, mas sem alterar sua interpretação.
Assim, não desejamos que uma aplicação do processo de
conversão-$\lambda$ elimine variáveis livres ou as introduza onde
antes elas não ocorriam.

Com isso em mente, considere a seguinte simplificação:

\begin{exe}
	\ex  $(\lambda m.\ \lambda n.\ m+n)(n)$\\
	$\lambda n.\ n+n$ \label{gg}
\end{exe}


\n Na primeira linha, temos a aplicação de uma função a um
argumento, que, neste caso, é uma ocorrência da variável
\textit{n}. Essa ocorrência aparece livre na expressão, a
qual, portanto, não tem um valor definido. A segunda linha é o
resultado da conversão-$\lambda$. Seguimos a receita que já
sabemos de cor: substituímos todas as ocorrências livres
da variável \textit{m} no corpo da função pelo argumento em
questão, no caso a variável \textit{n}. Entretanto, o resultado
foi uma expressão sem variáveis livres e, portanto, com um valor
definido. De fato, trata-se da função que leva um número natural
\textit{n} qualquer no seu dobro \textit{2n}. Onde está o
problema? Note que, ao substituir a variável \textit{m} por
\textit{n}, essa última acabou acidentalmente ligada pelo operador
lambda no corpo da função. É exatamente isso que precisamos
proibir. E não é só em casos em que o argumento da função é uma
variável que esse problema pode ocorrer. Casos em que o argumento
contém uma ocorrência livre de uma variável também são
suscetíveis ao problema, conforme mostra o exemplo abaixo:

\begin{exe}
	\ex $(\lambda m. \lambda n. m+n)(n+2)$\\
	$\lambda n. n+2+n$\\
	$\lambda n. 2n+2$\label{sl}
\end{exe}

\n Novamente, a conversão-$\lambda$ nos levou de uma expressão
contendo uma ocorrência livre de uma variável para outra
expressão sem variáveis livres. Para evitar essa consequência
indesejada, só se considera lícito o uso da conversão-$\lambda$ em
uma expressão formada por uma função $\lambda v. \gamma$
aplicada a um argumento \textit{a} que obedecer à seguinte
condição: se \textit{x} é uma ocorrência livre de uma
variável em \textit{a}, então a substituição das
ocorrências livres de \textit{v} pelo argumento \textit{a}
no corpo da função não pode resultar na ligação de \textit{x}
no corpo dessa função.

Como, então, simplificar a aplicação funcional em (\ref{gg})
via conversão-$\lambda$? A saída é usar a conversão-$\alpha$ antes
de usar a conversão-$\lambda$. Como vimos na seção inicial deste capítulo, a conversão-$\alpha$ é o processo de substituição da variável que segue o operador lambda e todas suas ocorrências no corpo de uma função por uma outra variável, sem alterar a natureza da função:\\

\n $(\lambda n.\ n + 1) \Leftrightarrow (\lambda m.\ m + 1)
\Leftrightarrow (\lambda k.\ k + 1)$\\

\n No caso com que estamos lidando, a conversão-$\alpha$ se
aplicará à subexpressão que aparece sublinhada abaixo,
substituindo-se \textit{n} por uma variável que ainda não apareça
na expressão, como \textit{k}, por
exemplo:\\

\begin{exe}
	\ex $(\lambda m.\ \underline{\lambda n.\ m+n})(n)$\\
	$(\lambda m.\ \lambda k.\ m+k)(n)$ \hfill
	Conversão-$\alpha$\\
	$\lambda k.\ n+k$ \hfill Conversão-$\lambda$
\end{exe}

\n Essa é, de fato, a grande utilidade da conversão-$\alpha$:
preparar uma expressão para a aplicação da conversão-$\lambda$.

Assim como a conversão-$\lambda$, a conversão-$\alpha$ também tem seus caprichos. Comecemos pela função abaixo:\\

\n $\lambda n.\ 5$\\

\n Note que se trata de uma função um tanto peculiar, já que não
há qualquer ocorrência da variável \textit{n} no corpo da
função. Isso significa que o valor retornado pela função não
depende da natureza do argumento a que ela se aplica. Qualquer que
seja esse argumento, o valor retornado será sempre o número 5. Uma
função como essa é chamada de \textsc{função constante}.
Funções de ordem superior também podem ser constantes, como
deixam claro os
exemplos abaixo:\\

\n $\lambda n.(\lambda m.\ m+1)$\\

\n $\lambda f.\ 3$\\

\n No primeiro caso, não importa o argumento: o valor retornado é
sempre a função sucessor. No segundo caso, qualquer que seja a
função que sirva de argumento, o valor retornado é sempre o
número 3.

Considere agora o exemplo abaixo:\\

\n $\lambda n.(\lambda n.\ n+1)$\\

\n Note que o corpo dessa função é a função sucessor. Se
formos descrever a função acima em palavras, diríamos tratar-se
de uma função que toma números naturais como argumentos e que
retorna a função sucessor como valor. Estamos diante de uma
função constante! Alguém poderia perguntar: mas a expressão
$n+1$ não contém uma ocorrência da variável \textit{n}? Sim,
mas trata-se de uma ocorrência \textit{ligada} por um
operador lambda no próprio corpo da função. A conclusão que
podemos tirar disso é que uma expressão do tipo $\lambda v.
\alpha$ representará uma função constante sempre que a variável
\textit{v} não aparecer \textit{livre} em $\alpha$.

A aplicação da função acima a um número qualquer --- 3, por
exemplo --- pode ser representada e simplificada por
conversão-$\lambda$ da
seguinte forma:\\

\n $(\lambda n.\ \lambda n. n+1)(3)$\\
$(\lambda n.\ n+1)$ \hfill (conversão-$\lambda$)\\

\n Com isso queremos deixar claro que, no processo de
conversão-$\lambda$ aplicado a uma expressão do tipo $(\lambda v.
\gamma)(a)$, apenas as \textit{ocorrências livres} de
\textit{v} em $\gamma$ devem ser substituídas por \textit{a}.

Ainda em conexão com o que estamos vendo, compare as duas
funções abaixo:

\begin{exe}
	\ex $(\lambda n.(\lambda m.\ m+n))$ \label{ki}
\end{exe}

\begin{exe}
	\ex $(\lambda m.(\lambda m.\ m+m))$ \label{al}
\end{exe}


\n Note que se trata de duas funções distintas. Isso fica
claro ao percebermos que a segunda, mas não a primeira, é uma
função constante. A questão que se coloca aqui é a seguinte: se
olharmos atentamente, veremos que a função em (\ref{al}) é o
resultado da aplicação cega do processo de conversão-$\alpha$ à
função em (\ref{ki}), ou seja, da substituição de todas as ocorrências de $n$ por $m$. Mas, quando introduzimos esse processo,
salientamos que sua aplicação à representação de uma função
preserva a identidade dessa função. O que então ocorreu de
errado no caso acima? Note que em (\ref{ki}) a variável \textit{n}
ocorria livre no corpo da função. Ao ser substituída por
\textit{m} na expressão $m+n$, a nova ocorrência de
\textit{m} acabou \textit{acidentalmente ligada} por um outro
operador lambda no corpo da nova função, que acabou se
transformando em uma função constante.

Para que o processo de conversão-$\alpha$ preserve sempre a
identidade de uma função, sua aplicação só é considerada
lícita se a variável que está sendo substituída não acabar
acidentalmente ligada no corpo da nova função.

Finalizamos aqui nossa discussão sobre esses os processos de conversão-$\lambda$ e conversão-$\alpha$,
complementando a breve introdução à notação lambda para a representação de funções apresentada no início do capítulo.

\bigskip

\begin{tcolorbox}[parbox=false,boxrule=0pt,sharp corners,breakable]


\section*{Sugestões de leitura}
\addcontentsline{toc}{section}{Sugestões de leitura}

A apresentação da notação lambda neste capítulo pode ser qualificada de instrumental,
já que o objetivo foi utilizar tal ferramenta na construção
do sistema interpretativo que começamos a desenvolver. Os leitores interessados em um tratamento formal mais
rigoroso, no qual a notação lambda é integrada a um dos sistemas
lógicos conhecidos como \textsc{cálculo lambda} devem consultar a
excelente introdução em \cite{carpenter97}, capítulo 2. Assumimos neste capítulo que predicados verbais têm como extensões funções que tomam indivíduos como argumentos. Para um tratamento diferente, baseado na ideia de que (alguns) predicados verbais tomam eventos como argumentos, ver \cite{davidson67}. Para extensões das ideias de Davidson, incorporando a noção de papel temático nas relações entre predicados e argumentos, ver \cite{parsons90} e \cite{schein93}. Ver também \cite{kratzer03} para um tratamento composicional.

\end{tcolorbox}


\bigskip

\begin{tcolorbox}[parbox=false,boxrule=0pt,sharp corners,breakable]

\section*{Exercícios}
\addcontentsline{toc}{section}{Exercícios}

\n\textbf{I.} Na seção sobre verbos bitransitivos, dissemos que as extensões abaixo eram equivalentes:\\

\n \den{apresentou} = $\lambda z.\ \lambda y.\
\lambda x.\ x\ \text{apresentou}\ z\ \text{para}\ y$\\

\n \den{apresentou} = $\lambda x.\ \lambda y.\
\lambda z.\ z\ \text{apresentou}\ x\ \text{para}\ y$\\

\n Calcule, passo a passo, as condições de verdade de \textit{João apresentou Pedro para Maria} usando cada uma delas e confira se os resultados são os mesmos. Em seguida, faça o mesmo com a seguinte extensão:\\

\n \den{apresentou} = $\lambda x.\ \lambda y.\
\lambda z.\ x\ \text{apresentou}\ y\ \text{para}\ z$\\

\n O resultado continua o mesmo? Por quê?\\

\n\textbf{II.} Já se propôs na literatura sintática que a estrutura argumental de verbos bitransitivos como \textit{apresentar} é, na verdade, algo como o seguinte (ver \cite{larson88} e, para uma apresentação em nível introdutório, \cite{carnie13}, capítulo 14):\\


\Tree [ [ João ].NP [ V$_{1}$ [ [ Pedro ].NP [ [ apresentou ].V$_{2}$  \qroof{para Maria}.PP ].V$_{2}^{\prime}$ 
].VP$_{2}$ ].VP$_{1}$ ].S

\bigskip

\n Nesse caso, uma operação sintática (sem reflexos semânticos) deslocaria o verbo para a posição V$_{1}$, logo após o sujeito, originando a ordem superficial S V OD OI. Assumindo que é a estrutura acima (sem movimento) que chega ao componente semântico, e que V$_{1}$ seja semanticamente vácuo, proponha uma entrada lexical para o verbo apresentar, prestando bastante atenção no que foi visto no exercício anterior.\\

\n\textbf{III.} Considere o par de sentenças abaixo:\\

\n (a) João viu Pedro.

\n (b) Pedro foi visto por João.\\

\n Para (b), assuma a seguinte estrutura:\\

\Tree [ [ [ Pedro ].N ].NP$_{1}$ [ [ foi-visto ].V [ [ por ].P [ [
João ].N ].NP$$_{2}$$ ].PP ].VP ].S

\n Para efeitos deste exercício, assuma que tanto \textit{viu}
quanto \textit{foi visto} são itens lexicais e que a preposição
\textit{por} é semanticamente vácua. Proponha então uma extensão
para \textit{viu} e outra para \textit{foi-visto} que dê conta do
fato de que as duas sentenças acima têm as mesmas
condições de verdade. Qual a relação entre essas duas
extensões que você acabou de propor?\\

\n\textbf{IV.} Considere a sentença abaixo:\\

\n (a) João elogiou-se.\\

\n Assuma para essa sentença a seguinte estrutura sintática:\\

\Tree [.S [.NP João ] [.VP [.V elogiou ] se ] ]

\bigskip

\n Assuma agora que o pronome reflexivo \textit{se} tenha a
seguinte extensão:\\

\n \den{se} = $(\lambda F.\lambda x.\ F(x)(x)=1)$\\

\n (i) Calcule as condições de verdade de (a) valendo-se da
entrada lexical acima e verifique se o resultado obtido é ou não
adequado.\\

\n (ii) Qual o tipo semântico da extensão acima? Qual o tipo do argumento que essa extensão toma? Qual o tipo do valor que ela retorna? Essa extensão
costuma ser chamada de função de reflexivização ou intransitivização. Pense no tipo de
transformação que ela efetua sobre seu argumento e diga o porquê
de ela ser chamada por esses nomes.\\

\n\textbf{V.} Neste capítulo, assumimos que o verbo de ligação
(cópula) em uma sentença como (a) é semanticamente vácuo:\\

\n (a) João é pianista\\

\n Entretanto, é possível atribuir a esse verbo uma extensão,
chamada de função identidade, que pode ser representada como em
(b) ou como em (c) abaixo:\\

\n (b) \den{é} = $\lambda f_{\langle e,t\rangle}.\ f $

\n (c) \den{é} = $\lambda f_{\langle e,t\rangle}.\lambda x_{e}.\
f(x)=1$\\

\n (i) Mostre que tanto (b) quanto (c) derivam condições de
verdade adequadas para (a).\\

\n (ii) Mostre que (b) e (c) são a mesma função.


\end{tcolorbox}













%%

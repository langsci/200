%versao de 23-MAR-2019

\chapter{Referência}

Neste capítulo, analisaremos expressões cujas extensões são
indivíduos. Chamaremos essas expressões de \textsc{expressões
referenciais}. Voltaremos a falar sobre nomes próprios, e
discutiremos descrições definidas e pronomes. Nosso léxico será
ampliado com a introdução do artigo definido singular
(\textit{o}, \textit{a}) e de pronomes pessoais como \textit{ele} e
\textit{ela}.

\section{Descrições definidas}

Descrições definidas são expressões introduzidas pelo artigo
definido (\textit{o}, \textit{a}, \textit{os}, \textit{as}) seguido de um sintagma nominal, como em
\textit{o livro}, \textit{a capital de Portugal}, \textit{os jogadores de futebol} e \textit{as praias do Brasil}. Do ponto de vista sintático, vamos
categorizar o artigo definido como sendo um determinante (D), que
toma como complemento um sintagma nominal (NP), sendo a projeção
resultante um sintagma determinante (DP):

\begin{figure}[H]
	\centerline{ \Tree [ {\ \ \ \ \ D\ \ \ \ \ }  {\ \ \ \ \ NP\ \ \ \ \ } ].DP } \caption{Sintagma determinante }
\end{figure}


\n NPs, a partir de agora, são estruturas nucleadas por um substantivo (nome), sem a presença do artigo. Nos exemplos do parágrafo anterior, são NPs as sequências \textit{livro}, \textit{capital de Portugal}, \textit{jogadores de futebol} e \textit{praias do Brasil}. Voltaremos aos nomes próprios mais adiante.

Passemos, então, ao significado de uma descrição definida.
Limitaremos nossa atenção às descrições definidas singulares,
comparando-as com o que já vimos sobre os nomes próprios (também
singulares). Assim como no caso dos nomes próprios, valemo-nos a
todo instante de descrições definidas para falarmos de objetos,
pessoas, lugares e eventos. Por exemplo, empregamos a descrição
definida \textit{o prefeito de Campinas} na sentença abaixo para falarmos
da pessoa que ocupa atualmente este cargo político.

\begin{exe}
\ex O prefeito de Campinas é paulista.\label{def}
\end{exe}

\n Se o atual prefeito de Campinas é João da Silva (JS), então (\ref{def})
é verdadeira se JS é paulista e falsa se JS não é paulista. Iremos
captar este fato, assumindo que a extensão de um DP definido é o
indivíduo que satisfaz a descrição apresentada pelo NP que segue o
artigo definido.

\begin{exe}
	\ex \den{[$_{\text{DP}}$ o prefeito de Campinas ]} = o indivíduo \textit{a}, tal que \textit{a} é prefeito de Campinas
\end{exe}

\n Qual a extensão do artigo definido \textit{o}? Olhemos para a
estrutura (simplificada) de (\ref{def}) dada na figura a seguir:

\begin{figure}[H]
	\centerline{ \Tree [ [ [ o ].D \qroof{prefeito de Campinas}.NP ].DP \qroof{é paulista}.VP ].S } \caption{Sentença com DP definido }
\end{figure}



\n A extensão do NP \textit{prefeito de Campinas} é uma função que leva
um indivíduo \textit{x} no valor de verdade 1, se \textit{x} é
prefeito de Campinas, e no valor de verdade 0, se \textit{x} não é
prefeito de Campinas. No cenário que estamos considerando, esta
função leva JS no valor 1 e todos os demais indivíduos no valor 0.
O que o artigo definido parece fazer, então, é olhar para a extensão
de NP e selecionar o indivíduo ao qual esta função atribui valor
1. Isso é o que está formalizado na entrada lexical abaixo (que
ainda sofrerá modificações):

\begin{exe}
	\ex \den{o} = $\lambda F_{\langle e,t\rangle}.\ \text{o indivíduo \textit{a}, tal que}\ F(a) = 1$
\end{exe}

\n Há uma notação compacta para representar o único indivíduo que satisfaz uma certa condição $\phi$, que é $\iota x[\phi]$. Valendo-nos dela, e da notação de predicados, podemos representar a extensão de \textit{o} da seguinte forma:

\begin{exe}
	\ex \den{o} = $\lambda F_{\langle e,t\rangle}.\ \iota x[F'(x)]$
\end{exe}

\n A extensão do artigo definido é, portanto, uma função de tipo
$\langle\langle e,t\rangle,e\rangle$. Como a extensão de NP é de
tipo $\langle e,t\rangle$, utilizamos aplicação funcional para
obter a extensão do DP sujeito em (\ref{def}), que será de tipo
\textit{e}.

\begin{exe}
	\ex \den{[$_{\text{DP}}$ o prefeito de Campinas ]} = \den{o}(\den{[$_{\text{NP}}$ prefeito de Campinas ]}) \\
		= $\iota x[\llbracket \text{NP} \rrbracket(x)=1]$ \\
		= $\iota x[\ \predica{prefeito}{x, Campinas}]$ \\
		= o indivíduo \textit{x}, tal que \textit{x} é prefeito de Campinas
\end{exe}

\n Como a extensão de VP é de tipo $\langle e,t\rangle$,
utilizando aplicação funcional mais uma vez, chegamos às condições
de verdade de (\ref{def}).

\begin{exe}
	\ex \den{(\ref{def})} = 1 \textit{sse} \textsc{paulista}$(\iota x[  \predica{prefeito}{x, Campinas}]) $ \\
		\den{(\ref{def})} = 1 \textit{sse} o indivíduo \textit{x}, tal que \textit{x} é prefeito de Campinas, é paulista
\end{exe}

\subsection{Unicidade}

\n As condições de verdade acima parecem adequadas. A sentença
(\ref{def}) será, de fato, verdadeira, se o indivíduo que ocupa
atualmente o cargo de prefeito de Campinas, seja ele quem for, for
paulista, e falsa se este indivíduo não for paulista. Nossa
entrada lexical para o artigo definido singular parece correta.

Consideremos agora as sentenças abaixo, e perguntemo-nos se elas
são verdadeiras ou falsas:

\begin{exe}
\ex O presidente de Campinas é paulista.\label{pre}
\ex O vereador de Campinas é paulista.\label{ver}
\end{exe}

Certamente, não queremos dizer que (\ref{pre}) e (\ref{ver}) são
verdadeiras. Seriam elas falsas? Mas, se esse fosse o caso,
preveríamos que as negações dessas sentenças fosses verdadeiras:

\begin{exe}
\ex O presidente de Campinas não é paulista.\label{nep}
\ex O vereador de Campinas não é paulista.\label{nev}
\end{exe}

Entretanto, compare essas sentenças com a contraparte negativa da
sentença (\ref{def}):

\begin{exe}
\ex O prefeito de Campinas não é paulista.\label{ner}
\end{exe}

Enquanto (\ref{ner}) é a coisa certa a se dizer se acreditamos que
(\ref{def}) seja falsa, o uso de (\ref{nep}) e (\ref{nev}) é no
mínimo estranho. Na verdade, o uso destas contrapartes negativas
parece tão inadequado quanto o uso de (\ref{pre}) e (\ref{ver}).


O que parece problemático com o uso das descrições definidas em
(\ref{pre})-(\ref{nev}) é que as descrições introduzidas pelos NPs
\textit{presidente de Campinas} e \textit{vereador de Campinas} não se aplicam a
um único indivíduo. Por um lado, não existe presidente de
Campinas, e por outro existem vários vereadores na cidade. E de
fato, uma pessoa que use (\ref{pre}) e (\ref{nep}) parece
pressupor, tomar como certo, que exista um presidente de Campinas.
Já uma pessoa que use (\ref{ver}) ou (\ref{nev}) parece tomar como
certo que Campinas possua apenas um vereador.

Em termos mais formais, as extensões dos NPs \textit{presidente de
Campinas} e \textit{vereador de Campinas} não retornam o valor de verdade
1 para um único indivíduo. No caso de \textit{presidente de Campinas}
todo indivíduo é levado ao valor de verdade 0, e no caso de
\textit{vereador de Campinas} mais de um indivíduo é levado ao valor de
verdade 1. O exemplo que havíamos discutido mais acima envolvendo
o NP \textit{prefeito de Campinas} não era problemático, justamente
porque a extensão deste NP retorna o valor 1 para um, e somente
um, indivíduo.

Para captar esses fatos, vamos reformular a entrada lexical do
artigo definido. Nossa entrada atual diz que o artigo definido é
uma função que toma como argumento funções de tipo $\langle
e,t\rangle$. Até aqui não impusemos nenhuma condição sobre essas
funções. É o que faremos agora. Vamos assumir que o domínio da
extensão do artigo definido contém apenas funções de tipo $\langle
e,t\rangle$ que retornam o valor 1 para um único indivíduo e 0
para todos os demais. Para representar esse tipo de função, vamos nos valer na metalinguagem da expressão $\exists !x$, que deve ser lida como `existe um único \textit{x}'. Vejamos:

\begin{exe}
	\ex \den{o} = $\lambda F_{\langle e,t\rangle}:\exists !x \in D_{e}[ F'(x)].\ \iota x[F'(x)]$
\end{exe}


\n A fórmula \underline{$\exists !x \in D_{e}[
F'(x)]$} que aparece na entrada acima indica uma
condição sobre o domínio da extensão do artigo definido. Dizemos
que essa extensão é uma \textsc{função parcial} de tipo
$\langle\langle e,t\rangle ,e\rangle$, já que seu domínio é um
subconjunto de D$_{\langle e,t\rangle}$. Para pertencer ao domínio
dessa função, não basta ser uma função de tipo $\langle
e,t\rangle$. É preciso ser uma função que retorne o valor 1 para
um único indivíduo.

Vejamos as consequências disso para o nosso sistema. Se voltarmos
a (\ref{pre})-(\ref{nev}), e tentarmos derivar a extensão de DP
utilizando aplicação funcional, encontraremos um problema, já que
a extensão de NP não pertence ao domínio da extensão do artigo
definido. Não havendo como continuar a derivação, o sistema não
atribui condições de verdade às sentenças. Podemos dizer
para esses casos que a existência de um único presidente ou
vereador de Campinas é condição necessária para que as respectivas
sentenças recebam um valor de verdade. No caso da sentença
(\ref{def}), que contém o DP \textit{prefeito de Campinas}, a condição de
que exista um único indivíduo que é prefeito de Campinas é
satisfeita e a sentença será verdadeira ou falsa, a depender se
este indivíduo é ou não paulista. Já nos casos de
(\ref{pre})-(\ref{nev}), as respectivas condições não são
satisfeitas e as sentenças não são verdadeiras nem falsas. Podemos
dizer que seu uso é infeliz, dada a inexistência de um único
presidente ou vereador de Campinas.

Condições como a condição de unicidade imposta pelo artigo
definido nos casos acima são chamadas de \textsc{pressuposições}.
Analisada do ponto de vista semântico, dizemos que uma sentença
\textit{S} pressupõe \textit{p} se a verdade de \textit{p} é
condição necessária para que \textit{S} seja verdadeira ou falsa.
Do ponto de vista pragmático, ou seja, do uso da linguagem,
dizemos que \textit{S} pressupõe \textit{p} se \textit{S} só pode
ser usada adequadamente em contextos em que a verdade de
\textit{p} é tomada como certa, ou seja, como conhecimento compartilhado pelo falante e por sua audiência. Em uma visão ainda mais estritamente pragmática, diz-se que um falante pressupõe \textit{p} quando ele age tomando como certa a verdade de \textit{p}.

Tendo isso em mente, não é difícil notar que a condição de
unicidade que introduzimos na entrada lexical do artigo definido
é, na maioria das vezes, excessivamente restritiva. Por exemplo,
imagine que estejamos envolvidos em uma conversa sobre a atual
conjuntura política da cidade de Campinas, e que no meio desta
conversa alguém diga o seguinte:

\begin{exe}
\ex O prefeito precisa melhorar o transporte municipal.\label{mun}
\end{exe}

\n De acordo com o que dissemos acima, deveríamos esperar que o
uso de (\ref{mun}) nesse contexto fosse infeliz, já que a mesma
pressupõe a existência de um único prefeito no mundo, e esse,
obviamente, não é o caso. Entretanto, o uso de (\ref{mun}) na
situação acima é perfeitamente natural. A razão é que o contexto
deixa claro que estamos falando do prefeito de Campinas. Vamos
dizer que o único indivíduo saliente no contexto que
ocupa o cargo de prefeito é o prefeito de Campinas. Se, ao
contrário, nossa conversa estivesse girando em torno de uma
reunião entre o prefeito de Campinas e o prefeito do Rio de
Janeiro, nossa sentença já não seria mais adequada, pois o
contexto, neste caso, torna salientes dois indivíduos que são
prefeitos. Precisamos então adicionar o conceito de saliência à
nossa entrada lexical do artigo definido:

\begin{exe}
	\ex \den{o} = $\lambda F_{\langle e,t\rangle}:\exists !x_{e}\ \textit{saliente no contexto, tal que}\ [F'(x)].\ \iota x[F'(x)\ \& $ \textit{x} está saliente no contexto $]$
\end{exe}

\n Especificar exatamente como um indivíduo se torna saliente em
um contexto é um dos tópicos mais controversos nos estudos da
interface Se\-mân\-ti\-ca-Prag\-má\-ti\-ca.
Con\-ten\-ta\-re\-mo-\-nos aqui em notar que o uso de descrições
definidas é, como acabamos de ver, sensível a esse conceito, e que
a entrada lexical acima busca representar, ainda que de maneira
rudimentar, essa sensibilidade. Salvo indicações em contrário, omitiremos essa importante ressalva no restante desse livro.



\section{Nomes próprios}


Temos agora dois tipos de expressões referenciais em nosso
sistema: nomes próprios e descrições definidas. Neste ponto, seria
interessante comparar as condições de verdade de pares de
sentenças cujos membros diferem apenas pelo uso de um nome próprio
no lugar de uma descrição definida, ou vice-versa. Comparemos, por
exemplo, as condições de verdade de (\ref{def}), repetida abaixo
como (\ref{defi}), com as condições de verdade de (\ref{js}):

\begin{exe}
\ex O prefeito de Campinas é paulista.\\
\den{(\ref{defi})} = 1 \textit{sse} o indivíduo \textit{a}, tal
que \textit{a} é prefeito de Campinas é paulista\label{defi}
\end{exe}

\begin{exe}
\ex João da Silva é paulista.\\
\den{(\ref{js})} = 1 \textit{sse} JS é paulista.\label{js}
\end{exe}


\n Observe a diferença entre o que está à direita de \textit{sse}
em (\ref{defi}) e (\ref{js}). Esta diferença reflete uma diferença
de significado entre as sentenças. Não queremos atribuir a
(\ref{defi}) e (\ref{js}) o mesmo significado, ou seja, as mesmas
condições de verdade, já que podemos facilmente imaginar situações
em que uma é verdadeira e a outra falsa. Basta que JS não seja
prefeito de Campinas. E, mesmo que JS seja prefeito de Campinas,
uma pessoa que não esteja ciente desse fato pode muito bem
acreditar que (\ref{defi}) seja verdadeira e (\ref{js}) falsa, ou
vice-versa, sem que atribuamos a esta pessoa crenças
contraditórias. O que o nosso sistema diz sobre as situações em
que JS é, de fato, prefeito de Campinas, é que (\ref{defi}) e
(\ref{js}) têm a mesma extensão, o valor de verdade 1, não o mesmo
significado. Obviamente, na origem dessa distinção está a
diferença na maneira como apresentamos a extensão do nome próprio
João da Silva e a extensão do DP \textit{o prefeito de Campinas}:

\begin{exe}
	\ex \den{João da Silva} = JS
\end{exe}

\begin{exe}
	\ex \den{[o prefeito de Campinas]} = o indivíduo \textit{a}, tal que \textit{a} é prefeito de Campinas
\end{exe}

\n Mesmo que, em certas situações, \textit{João da Silva} e \textit{o prefeito de Campinas} tenham a mesma extensão (o indivíduo JS), eles não têm o
mesmo significado, e isto está expresso na diferença que acabamos
de enfatizar. A extensão do nome próprio é identificada
diretamente com o indivíduo JS, enquanto que a extensão da
descrição definida o é através, como o próprio nome já diz, da
descrição de uma das propriedades do indivíduo. Com isso,
diferentes situações ou estados de coisas podem levar a diferentes
extensões para a descrição definida \textit{o prefeito de Campinas}. Se
Pedro de Oliveira, em vez de de João da Silva, for o prefeito de
Campinas, então a extensão da descrição definida será o indivíduo
Pedro de Oliveira, em vez do indivíduo João da Silva. Já no caso
do nome próprio \textit{João da Silva}, diferentes situações ou estados
de coisas não alteram sua extensão, que é sempre o mesmo indivíduo
João da Silva. Uma expressão cuja extensão é fixa, ou seja, que
nunca varia de situação para situação, é chamada de
\textsc{designador rígido}. Assim, em nosso sistema, nomes
próprios, mas não descrições definidas, são designadores rígidos.

Evidência para essa distinção entre nomes próprios
(designadores rígidos) e descrições definidas
(designadores não rígidos) pode ser obtida através do
seguinte exercício mental: para cada uma das sentenças abaixo,
imaginemos diferentes situações em que a sentença seja verdadeira.

\begin{exe}
\ex O autor de Dom Casmurro era carioca.\label{perv}
\ex Machado de Assis era carioca.\label{lul}
\end{exe}

No caso de (\ref{perv}), nossa imaginação pode nos levar a
situações em que, por exemplo, José de Alencar tenha nascido no
Rio de Janeiro e seja o autor de Dom Casmurro. Ou podemos pensar
que Paulo Coelho tenha escrito Dom Casmurro, e que era um carioca
do século dezenove. Pouco importa quem imaginamos como sendo o
autor de Dom Casmurro. O que há em comum entre essas situações é
que o autor de Dom Casmurro, seja ele quem for, nasceu no Rio de
Janeiro. Já no caso de (\ref{lul}), todas as situações que
imaginamos dizem respeito ao mesmo indivíduo: Machado de Assis.
Seja ele escritor ou não, pouco importa. Mas note que temos a
intuição de que é sempre do mesmo indivíduo que estamos falando. A
ideia é que essa diferença, nas situações hipotéticas que
construímos ao imaginar (\ref{perv}) e (\ref{lul}) como verdadeiras,
é reflexo do fato de a descrição definida \textit{o autor de Dom
Casmurro} não ser um designador rígido, enquanto o nome
próprio \textit{Machado de Assis} o é.

Apesar da evidência acima a favor do tratamento dos nomes próprios
como designadores rígidos, a análise não escapa de um problema que
foi objeto da atenção de vários filósofos e linguistas. Para
entendermos a fonte do problema, é preciso nos darmos conta da
possibilidade de uma pessoa possuir mais de um nome. Por exemplo,
a famosa atriz brasileira conhecida pelo nome de Fernanda
Montenegro, chama-se na verdade Arlette Torres (seu nome de
casada). Muitas pessoas, entretanto, não sabem disso. Imagine,
agora, uma pessoa que tenha conhecido Fernanda/Arlette antes de ela
se tornar atriz e adotar o nome artístico pelo qual é conhecida.
Imagine que essa pessoa tenha visto Fernanda/Arlette pela última
vez quando ela ainda era uma criança e que, mesmo tendo visto a atriz atuar nas novelas, jamais tenha se dado conta de que Arlette Torres e
Fernanda Montenegro são a mesma pessoa. Por fim, imagine que, num
belo dia, alguém chegue para essa pessoa e diga a sentença abaixo:

\begin{exe}
\ex Arlette Torres é Fernanda Montenegro.\label{fer}
\end{exe}

\n A pessoa em questão certamente se surpreenderia com essa
revelação. Até aquele momento, tal pessoa provavelmente acreditava
na negação de (\ref{fer}), ou seja no conteúdo de (\ref{fre}):

\begin{exe}
\ex Arlette Torres não é Fernanda Montenegro.\label{fre}
\end{exe}

Mas vejamos que significado, ou seja, que condições de verdade, o
nosso sistema atribui a essas sentenças. Não dissemos nada sobre o
uso do verbo \textit{ser} em sentenças desse tipo, mas é natural
que sua extensão estabeleça a identidade entre as extensões dos
nomes presentes na sentença:

\begin{exe}
	\ex \den{é} = $(\lambda x.\ \lambda y.\ x = y)$
\end{exe}

\n E quanto aos nomes próprios? Como ambos os nomes são nomes da
mesma pessoa, não temos muita escolha:

\begin{exe}
	\ex \den{Arlette Torres} = \den{Fernanda Montenegro} = \textit{arlette torres}
\end{exe}

\n Lembre-se que o que aparece representado em itálico depois das igualdades é o indivíduo
(carne e osso). Assim, poderíamos alternativamente dizer:

\begin{exe}
	\ex \den{Arlette Torres} = \den{Fernanda Montenegro} = \textit{fernanda montenegro}
\end{exe}

\n Efetuando a derivação das condições de verdade de (\ref{fer})
(a derivação fica a cargo do leitor), obteremos o seguinte:

\begin{exe}
	\ex \den{(\ref{fer})} = 1 \textit{sse} \textit{arlette torres} = \textit{arlette torres}
\end{exe}

\n Mas o que aparece após a igualdade é uma tautologia, já que
sabemos de antemão que todo indivíduo é idêntico a si mesmo. Se
pensarmos melhor, essas são as mesmas condições de verdade que o
nosso sistema atribui à sentença abaixo:

\begin{exe}
	\ex Arlete Torres é Arlete Torres.\label{fli}
\end{exe}

\n Essa sentença, de fato, soa como tautológica, e certamente
ninguém, incluindo o personagem de nossa história, se
surpreenderia ao ouvi-la.  Como então explicar a surpresa de
nossa personagem diante de (\ref{fer}) e sua indiferença diante de
(\ref{fli}), se atribuímos a elas o mesmo significado?

Vendo o problema de outro ângulo: como nossa personagem
acredita na verdade de (\ref{fre}), e como nosso sistema atribui a
essa sentença as condições de verdade abaixo, deveríamos atribuir
a essa pessoa uma crença contraditória, já que
tais condições de verdade não são nunca satisfeitas:

\begin{exe}
	\ex \den{(\ref{fre})} = 1 \textit{sse} \textit{arlette torres} $\neq$ \textit{arlette torres}
\end{exe}

\n Mas acreditar em (\ref{fre}) certamente não implica
contradição. Estamos, pois, diante de um impasse gerado pela nossa
hipótese de que o valor semântico de um nome próprio é diretamente
identificado com o indivíduo portador do nome.

Não iremos investigar aqui possíveis reformulações em nosso
sistema para tentar lidar com esse problema. O leitor, entretanto,
deve estar ciente disso, bem como entrar em contato com a
literatura clássica sobre o tema (ver, para isso, as referências
ao final do capítulo). 



\subsection{Nomes próprios como predicados} 

\n Em todos os nossos exemplos até o momento, os nomes próprios
diferem sintaticamente das descrições definidas pelo fato de não
virem precedidos de um artigo definido. Isso, entretanto, não é
sempre verdade em português, conforme atestam as sentenças abaixo:

%xl
\begin{exe}
\ex\label{oj}
\begin{xlist}
\ex O João está sorrindo.\label{oja}
\ex O João beijou a Maria.\label{ojb}
\end{xlist}
\end{exe}

Casos como esses existem também em dialetos do grego e do alemão, por exemplo. A semelhança
sintática entre os nomes próprios em (\ref{oj}) e as descrições
definidas que analisamos mais acima parece clara o bastante para
que tentemos assimilar essas categorias. Considere, então, a
seguinte estrutura para o sintagma \textit{o João}:



\begin{figure}[H]
	\centerline{ \Tree [ [ o ].D [ [ João ].N ].NP ].DP } \caption{Nome próprio precedido por artigo definido }
\end{figure}

Considere, ainda, a possibilidade de estender esse tratamento aos casos em que o nome
pró\-prio não vem precedido do artigo. Para tanto, assumamos
que exista um determinante nulo, idêntico semanticamente ao artigo definido, mas
que não é realizado foneticamente, conforme ilustrado
abaixo:



\begin{figure}[H]
	\centerline{ \Tree [ [ $\emptyset$ ].D [ [ João ].N ].NP ].DP } \caption{Nome próprio com determinante nulo }
\end{figure}

\n Unificaríamos assim a estrutura sintática de constituintes como \textit{o
João} e \textit{o menino}, ambos tratados agora como DPs ramificados em D
e NP. Criamos, entretanto, um problema para o componente semântico.
Nossa entrada lexical para \textit{João} atribui a esse item uma extensão
de tipo e. Como a extensão do artigo definido é de tipo
$\langle\langle e,t\rangle ,e\rangle$, temos uma incompatibilidade
de tipos, impossibilitando a obtenção de um extensão para o DP \textit{o João}. Como podemos resolver esse impasse?

Tratamos NPs como \textit{menino} ou \textit{prefeito de Campinas} como
predicados com extensões do tipo $\langle e,t \rangle$. A solução
mais óbvia para o problema acima é tratarmos nomes próprios também
como sendo de tipo $\langle e,t \rangle$. Mas que função um nome
como \textit{João} teria como extensão? Uma possibilidade é assumir que a
extensão de \textit{João} seja uma função que leva um indivíduo
\textit{x} ao valor de verdade 1 se, e somente se, \textit{x} se
chama João. A definição exata de como um indivíduo vem a se
chamar João não nos interessa aqui. Batismo, registro em cartório,
título de nobreza e criação de apelido são apenas algumas das
maneiras por meio das quais alguém vem a ser chamado pelo nome que
tem. Como falantes de uma língua, sabemos que nomes próprios são
usados quando queremos nos referir a uma pessoa que passou por uma
dessas, digamos, cerimônias, tendo ali sua pessoa associada a uma
expressão verbal. É isso que queremos dizer com a condição `x se
chama NN' onde NN é um nome próprio qualquer. De acordo com essa
ideia, a entrada lexical de \textit{João} passa a ser a seguinte:

\begin{exe}
	\ex \den{João} = $\lambda x.\ x\ \text{se chama \textit{João}}$
\end{exe}

\n Uma vez mais, aplicação funcional nos dá a extensão de \textit{o João} a partir de seus constituintes imediatos, conforme mostrado abaixo:

\begin{exe}
	\ex ($\lambda F_{\langle e,t\rangle}:\exists !x\ \text{saliente no contexto}\ [F'(x)].\ \iota x[F'(x)]$)(\den{[$_{\text{NP}}$ João]}) \\
	= $\iota x[x\ \text{está saliente no contexto}\ \&\ x\ \text{se chama \textit{João}}]$
\end{exe}

\n Para obtermos as condições de verdade da sentença (\ref{oja}),
utilizamos aplicação funcional novamente. Desta vez, a extensão de
DP, tipo e, serve como argumento para a extensão de VP, tipo
$\langle e,t\rangle$.

\begin{exe}
	\ex \den{S} = \den{[$_{\text{VP}}$ está sorrindo]}([$_{\text{DP}}$ o João]) \\
	\den{S} = 1 \textit{sse} o único indivíduo saliente no contexto que se chama João está sorrindo.
\end{exe}

\n As condições acima parecem adequadas. Note a menção à saliência
contextual, semelhante ao caso das descrições definidas que
discutimos anteriormente. Podemos apreciar o efeito dessa condição
imaginando contextos que tornam salientes mais de um indivíduo com
o mesmo nome. Por exemplo, imagine que tenhamos dois amigos
comuns, ambos chamados João, e que os dois estejam caminhando em
nossa direção, mas apenas um deles esteja sorrindo. Em uma
situação como essa, o uso do nome próprio desacompanhado de
modificadores, como em \textit{o João está sorrindo}, não é adequado.
Nossa análise capta esse fato corretamente. Se sabemos que um
dos amigos é carioca e o outro paulista, e que este último é o que
está sorrindo, podemos dizer algo como \textit{O João paulista está
sorrindo}.

No que diz respeito à condição de unicidade, essa alternativa teórica capta corretamente a semelhança entre nomes próprios e descrições definidas. No entanto, perde-se a distinção importante entre a rigidez dos nomes e a não rigidez das descrições definidas. Não é difícil imaginar
diferentes situações em que os indivíduos chamados \textit{João}
não são os mesmos. A consequência disso é que a extensão dos DPs
\textit{João} ou \textit{O João} passa a variar de situação para situação. Em
outras palavras, nomes próprios não são mais designadores rígidos
de acordo com essa teoria, o que é problemático, dado o que vimos
anteriormente. Vamos aqui deixar em aberto a questão de como reparar
essa alternativa teórica de modo a tornar os nomes próprios
designadores rígidos.

Consideremos uma outra possibilidade. Tomemos a extensão
do nome próprio \textit{João} como sendo a função que leva o indivíduo
João no valor de verdade 1 e todos os demais
indivíduos no valor de verdade 0:

\begin{exe}
	\ex \den{João} = $(\lambda x.\ x=\text{\textit{joão}})$
\end{exe}

\n Para o DP \textit{O João}, obteríamos o seguinte:

\begin{exe}
	\ex \den{O João} = o único indivíduo \textit{x}, tal que \textit{x} = \textit{joão}
\end{exe}

\n Como ninguém além do próprio João pode satisfazer a condição acima, não é difícil notar que isso é equivalente ao que tínhamos
antes:

\begin{exe}
	\ex \den{O João} = \textit{joão}
\end{exe}

\n Dois pontos importantes: em primeiro lugar, como o único indivíduo
idêntico ao João é o próprio João, a condição de unicidade que
associamos ao artigo definido será sempre satisfeita. Em segundo
lugar, como a função correspondente à extensão do nome próprio é
definida mencionando diretamente o único indivíduo que ela leva ao
valor de verdade 1, os DPs em questão são designadores rígidos,
diferentemente dos DPs que não envolvem nomes próprios. \textit{João} ou
\textit{O João} denotam o indivíduo João qualquer que seja a situação em
questão. Preservam-se assim tanto as propriedades semânticas
associadas aos nomes próprios quanto aquelas associadas às
descrições definidas usuais que discutimos mais acima.

Vamos encerrar esse tema por aqui. Como se pode notar, nem mesmo os nomes próprios, que pareciam ser as expressões mais simples semanticamente, estão livres de controvérsias. Pelo contrário, como já salientamos, eles foram e continuam sendo alvos de debates lógicos, linguísticos e filosóficos. Mais uma vez, remetemos o leitor às sugestões de leitura ao final do capítulo.



\section{Pronomes}

Imagine que alguém nos apresente a sentença em (\ref{pro}) e nos
pergunte se ela é verdadeira ou falsa:

\begin{exe}
\ex Ele é italiano.\label{pro}
\end{exe}

\n Nossa reação, claro, será de perplexidade e, provavelmente, indagaremos algo como ``Mas ele quem?''. Se a pessoa nos disser
que está falando do João, ou se ela simplesmente apontar para
o João, assumindo que ele esteja nas proximidades, então diremos
que a sentença é verdadeira se o João for italiano e falsa se ele
não for. Tivesse a pessoa apontado para o Pedro e diríamos que a
sentença seria verdadeira se o Pedro fosse italiano e falsa se o
Pedro não fosse italiano.

Imagine agora que nos fosse apresentada a sentença em (\ref{dos}):

\begin{exe}
\ex Ele é tio dele.\label{dos}
\end{exe}

Dita, assim, fora de contexto, tal sentença também levaria a
indagações, talvez algo como ``Espere aí, quem é tio de quem?''.
Se a pessoa então repetisse a sentença, mas desta vez apontando
(enquanto fala) primeiro para o João e logo depois para o Pedro,
diríamos que a sentença seria verdadeira se o João fosse tio do
Pedro e falsa se o João não fosse tio do Pedro. Tivesse a pessoa
apontado primeiro para o Pedro e depois para o João, diríamos que
a sentença seria verdadeira se o Pedro fosse tio do João e falsa
se o Pedro não fosse tio do João.

Toda essa discussão está, obviamente, centrada  no uso do pronome
pessoal \textit{ele}. O que acabamos de ver já basta para tirarmos
algumas conclusões a esse respeito. Vamos listá-las:

\begin{itemize}

\item A exemplo dos nomes próprios e das descrições definidas, pronomes são usados a todo instante para falar de indivíduos (pessoas, animais, objetos, etc.).

\item Fora de contexto, não faz sentido perguntar se uma
    sentença que contém um pronome é verdadeira ou falsa. É só
    quando estamos cientes do contexto em que a sentença foi
    dita que podemos nos pronunciar a respeito.

\item Da mesma forma, não faz sentido perguntar a quem ou a que
    um pronome se refere, se não dermos informações sobre o
    contexto. Diferentes contextos levam a diferentes
    respostas.

\item Sentenças podem conter mais de um pronome e, nesse caso,
    o contexto de fala deve ser elaborado de modo a deixar
    claro a quem ou a que estamos nos referindo ao usar cada
    um dos pronomes. Mesmo dentro de uma mesma sentença,
    diferentes pronomes podem estar relacionados a diferentes
    indivíduos.

\end{itemize}


Refletindo sobre essa lista, podemos extrair algumas direções
teóricas sobre a semântica dos pronomes:

\begin{itemize}

\item Parece natural assumir que, a exemplo dos nomes próprios
    e das descrições definidas, a extensão de um pronome é um
    indivíduo. Pronomes seriam, portanto, expressões
    referenciais associadas ao tipo semântico \textit{e}.

\item No entanto, é preciso relativizar a extensão de um
    pronome a um parâmetro que traduza de alguma forma o papel
    do contexto de fala.

\item Da mesma forma, é preciso relativizar a extensão das
    sentenças que contêm pronomes a esse mesmo parâmetro.

\item É preciso marcar os diferentes pronomes que podem
    aparecer em uma sentença de modo a deixar claro se eles se
    referem ou não ao mesmo indivíduo.


\end{itemize}


Para distinguirmos as diferentes instâncias pronominais no
interior de uma mesma sentença, vamos fazer uso de índices
numéricos. Assim, nossas sentenças acima serão representadas da
seguinte forma:

\begin{exe}
\ex Ele$_{1}$ é italiano.\label{ita}
\ex Ele$_{1}$ é tio dele$_{2}$.\label{tio}
\end{exe}

Em (\ref{tio}), o uso de índices diferentes indica que os pronomes
têm referentes distintos. Já na sentença a seguir, os dois pronomes
têm o mesmo referente, já que são marcados com o mesmo índice:

\begin{exe}
\ex O tio dele$_{1}$ deu um carro pra ele$_{1}$.
\end{exe}

Para explicitarmos o papel do contexto e, com isso, especificarmos a extensão de um pronome, bem como as condições de verdade de uma sentença em que ele apareça, vamos nos valer de um conceito da lógica: a \textsc{atribuição}. Uma atribuição é uma função que leva números naturais em elementos de um certo domínio semântico. No nosso caso, esse domínio é $D_{e}$, o domínio dos indivíduos. Neste livro, vamos seguir uma implementação proposta em \cite{heikra98} e tratar as atribuições como \textsc{funções parciais} que têm como domínios subconjuntos dos números naturais. Todas as funções abaixo, por exemplo, são atribuições:

\begin{exe}
	
	\ex a. $\left[%
	\begin{array}{@{}c@{\ }l@{\ }l@{}}%
	1 & \rightarrow& \text{João} \\
	\end{array}\right]$
	\ \ \ \ \ b. $\left[%
	\begin{array}{@{}c@{\ }l@{\ }l@{}}%
	1 & \rightarrow& \text{João} \\
	2 & \rightarrow & \text{Pedro} \\
	\end{array}\right]$
	\ \ \ \ \ c. $\left[%
	\begin{array}{@{}c@{\ }l@{\ }l@{}}%
	1 & \rightarrow& \text{João} \\
	2 & \rightarrow & \text{Pedro} \\
	3 & \rightarrow & \text{Maria} \\
	\end{array}\right]$
	
	\label{as}
\end{exe}


Atribuições, junto com os índices, nos permitem
formalizar a variabilidade intrínseca das extensões dos pronomes. Assim, abre-se a possibilidade de captar parte do papel do contexto de fala na determinação dessas extensões e das extensões dos constituintes que delas dependem. De maneira geral, como é costume nos livros de lógica,
iremos nos referir a atribuições pela letra \textit{g} usada
como um expoente do símbolo de extensão: \den{$\alpha$}$^{g}$.
A seguir, estão alguns exemplos de entradas lexicais para os
pronomes:

\begin{exe}
\ex \den{ele$_{1}$}$^{g}$ = g(1)\label{g1}

\ex \den{ele$_{2}$}$^{g}$ = g(2)

\ex \den{ela$_{3}$}$^{g}$ = g(3)

\end{exe}

\n Generalizando, essas entradas lexicais devem ser lidas da
seguinte forma: para qualquer atribuição \textit{g} e qualquer
número natural \textit{i}, a extensão de ele$_{i}$ ou ela$_{i}$ é o
valor que \textit{g} atribui a \textit{i}, ou seja \textit{g(i)}. Por exemplo, se \textit{g} for a atribuição em (\ref{as}c), então temos que:

\begin{exe}

\ex \den{ele$_{1}$}$^{g}$ = João

\ex \den{ele$_{2}$}$^{g}$ = Pedro

\ex \den{ela$_{3}$}$^{g}$ = Maria

\end{exe}

Já para sentenças contendo pronomes, o que teremos é o seguinte:

\begin{exe}

\ex \den{ele$_{1}$ é italiano}$^{g}$ = 1 \textit{sse} João é italiano

\ex \den{ele$_{2}$ é italiano}$^{g}$ = 1 \textit{sse} Pedro é italiano

\ex \den{ela$_{3}$ gosta dele$_{1}$}$^{g}$ = 1 \textit{sse} Maria gosta de
João

\ex \den{ela$_{3}$ gosta dele$_{2}$}^{g} = 1 \textit{sse} Maria gosta de
Pedro

\end{exe}

Mas voltemos agora ao nosso cenário inicial em que a sentença
(\ref{pro}), repetida abaixo como (\ref{cla}), foi enunciada sem a
presença de um indivíduo saliente no contexto.

\begin{exe}
\ex Ele é italiano.\label{cla}
\end{exe}

Como notamos, tal uso é inadequado, ficando no ar a pergunta: a
quem o pronome se refere? Continuando na esteira do que foi proposto em \cite{heikra98}, podemos associar
tais usos a casos em que o índice de um pronome não pertence ao
domínio da atribuição. Para tornar isso mais explícito, façamos
uma pequena modificação no nosso esquema
correspondente à entrada lexical dos pronomes:

\begin{exe}
	\ex Entrada lexical dos pronomes\\
	Para qualquer pronome \textit{pron}, qualquer atribuição
	\textit{g}, e qualquer número natural \textit{i},\\
	
	$\llbracket pron_{i}\rrbracket^{g} =
	\begin{cases}
	g(i) & \text{se } i\in\ \textit{Domínio de}\ g\\
	indefinido & \text{se } i\not\in\ \textit{Domínio de}\ g
	\end{cases}$
\end{exe}



\n Em palavras: a extensão de $pron_{i}$ em relação a \textit{g} é
o valor que \textit{g} atribui a \textit{i}, se \textit{i}
pertence ao domínio de \textit{g}, e indefinida, se \textit{i} não
pertence ao domínio de \textit{g}.

Para formalizar o caso especial em que simplesmente não há
indivíduos salientes no contexto, podemos pensar em uma
atribuição nula, que é uma função cujo domínio é o conjunto
vazio. Representaremos
essa atribuição por $\varnothing$. Temos então o seguinte:

%xl
\begin{exe}
	\ex
	\begin{xlist}
		\ex \den{ele$_{1}$}\gone{1}{João} = João
		\ex \den{ele$_{1}$}\gone{1}{Pedro} = Pedro
		\ex \den{ele$_{2}$}\gone{1}{João} = \textit{indefinido}
		\ex \den{ele$_{1}$}\gvaz = \textit{indefinido}  
	\end{xlist}
\end{exe}

\n No nível sentencial, temos o seguinte:

%xl
\begin{exe}
	\ex
	\begin{xlist}
		\ex \den{Ele$_{1}$ é italiano}\gone{1}{João} = 1 \textit{sse} João é italiano
		\ex \den{Ele$_{1}$ é italiano}\gone{1}{Pedro} = 1 \textit{sse} Pedro é italiano
		\ex \den{Ele$_{1}$ é italiano}\gvaz = \textit{indefinido}  
	\end{xlist}
\end{exe}


Para casos envolvendo mais de um pronome, a sentença ficará sem
valor de verdade sempre que um ou mais índices não pertencerem ao
domínio da atribuição, correspondendo a situações de fala em
que um número insuficiente de indivíduos se tornaram salientes:

%xl
\begin{exe}
	\ex
	\begin{xlist}
		\ex \den{Ele$_{1}$ é tio dele$_{2}$}\gtwo{1}{João}{2}{Pedro} = 1 \textit{sse} João é tio de Pedro
		\ex \den{Ele$_{1}$ é tio dele$_{2}$}\gone{1}{João} = \textit{indefinido}
		\ex \den{Ele$_{1}$ é tio dele$_{2}$}\gvaz = \textit{indefinido}  
	\end{xlist}
\end{exe}

Note que a entrada lexical dos pronomes não faz menção a seus traços gramaticais (gênero, número, pessoa). Isso, claro, não parece semanticamente correto. Afinal de contas, \textit{ele} e \textit{ela} não parecem sinônimos. Um falante competente sabe que, diante de um homem, por exemplo, usa-se \textit{ele}, ao passo, que diante de uma mulher, usa-se \textit{ela}. Mas nem sempre há uma relação transparente entre morfologia e semântica. Uma sentença como \textit{João chamou uma pessoa, mas ela não escutou} não fornece indícios sobre o referente do pronome \textit{ela} ser um homem ou uma mulher. Nesse caso, é o gênero gramatical do antecedente do pronome (o sintagma nominal \textit{uma pessoa}) que parece determinar a escolha da forma feminina \textit{ela}. No que segue, continuaremos simplificando, ignorando a relação de gênero entre pronomes e seus antecedentes e/ou referentes.

Por fim, uma questão técnica. Nossa entrada lexical para os
pronomes faz menção explícita a atribuições. Para que não só a
extensão dos pronomes, mas também a dos constituintes que os
contenham, sejam sensíveis a atribuições, precisamos reformular
nossos princípios composicionais de modo a fazer com que
constituintes herdem essa sensibilidade de seus constituintes
imediatos. Nossos dois princípios composicionais, o princípio dos
nós não ramificados e o princípio de
aplicação funcional, passam a ter as seguintes versões:


\begin{exe}
	\ex Princípio dos nós não ramificados \\
	Se $\alpha$ é um nó não ramificado cujo único constituinte imediato é $\beta$, então para qualquer atribuição \textit{g}, \den{$\alpha$}$^{g}$ = \den{$\beta$}$^{g}$.
\end{exe}

\begin{exe}
	\ex Aplicação funcional\\
	Seja $\alpha$ um nó ramificado, cujos constituintes imediatos são $\beta$ e $\gamma$. Para qualquer atribuição \textit{g}, se \den{$\beta$}$^{g}$ é uma função e \den{$\gamma$}$^{g}$ pertence ao domínio de \den{$\beta$}$^{g}$, então \den{$\alpha$}$^{g}$ = \den{$\beta$}$^{g}$(\den{$\gamma$}$^{g}$).
\end{exe}

Para que esses princípios se apliquem adequadamente, vamos assumir
que as extensões de todos os itens lexicais são relativizadas a atribuições. Assim, para qualquer atribuição $g$, temos que:

%xl
\begin{exe}
\ex\label{dr}
\begin{xlist}
\ex \den{ele$_{1}$}$^{g}$ = $g(1)$\label{dru}
\ex \den{João}$^{g}$ = \textit{joão}\label{dri}
\ex \den{ama}$^{g}$ = $\lambda x.\lambda y.\ \predica{ama}{y,x}$\label{drov}
\end{xlist}
\end{exe}

Relativizar todas as entradas lexicais a atribuições pode
parecer pôr em risco as distinções que salientamos acima entre
pronomes e nomes próprios, por exemplo. Mas esse não é o caso.
Note uma diferença crucial entre as extensões em (\ref{dru}) e
(\ref{dri}). Enquanto a primeira varia de atribuição para
atribuição, a segunda não, já que a variável \textit{g}
não aparece do lado direito da igualdade. Ou seja, para qualquer
atribuição \textit{g}, a extensão do nome \textit{João} (ou do
verbo \textit{amar} em (\ref{drov})) será a mesma. Com pronomes, entretanto, a
situação é diferente, conforme discutimos acima. Por isso, pronomes são frequentemente chamados de \textsc{variáveis}, em oposição a nomes próprios, chamados de \textsc{constantes} (termos emprestados da lógica de predicados). As distinções que
discutimos anteriormente ficam, portanto, preservadas. Para efeitos de
simplicidade, costuma-se omitir o expoente \textit{g} quando a
extensão em questão é insensível à natureza da atribuição. Esse
é o caso de todos os nomes próprios, descrições definidas e
predicados que havíamos analisado até aqui.

Vamos encerrar interpretando, passo a passo, uma sentença contendo
dois pronomes, cada um com um índice diferente. Utilizaremos
todo o  material que introduzimos nesta seção. A estrutura
sintática da sentença está representada na árvore logo a seguir. Assumiremos aqui que pronomes pessoais, como descrições definidas, são DPs. Os pronomes, entretanto, não tomam um NP complemento, sendo assim DPs não ramificados:

\begin{exe}
	\ex Ele$_{1}$ gosta dela$_{2}$\label{las}
\end{exe}

\begin{figure}[H]
	\centerline{ \Tree [ [ ele$_{1}$ ].DP [ [ gosta ].V [ [ de ].P [ ela$_{2}$ ].DP ].PP ].VP ].S } \caption{Pronomes como DPs }
\end{figure}

\begin{exe}
	\ex Derivação semântica de \textit{Ele}$_{1}$ \textit{gosta dela}$_{2}$: \\
	1. \den{ela$_{2}$}$^{g}$ = $g(2)$ \hfill (L) \\
	2. \den{PP}$^{g}$ = \den{DP}$^{g}$ = \den{ela$_{2}$}$^{g}$ = $g(2)$\hfill (1, NR, \textit{de} é vácua) \\
	3. \den{gosta}$^{g}$ = $\lambda x.\lambda y.\ \predica{gosta}{y,x}$ \hfill (L) \\
	4. \den{V}$^{g}$ = \den{gosta}$^{g}$ = $\lambda x.\lambda y.\ \predica{gosta}{y,x}$\hfill (3, NR) \\
	5. \den{VP}$^{g}$ = \den{V}$^{g}$(\den{PP}$^{g}$) \hfill (AF) \\
	6. \den{VP}$^{g}$ = $(\lambda x.\lambda y.\ \predica{gosta}{y,x})(g(2))$\hfill (2, 4, 5) \\
	7. \den{VP}$^{g}$ = $\lambda y.\ \ \predica{gosta}{y,g(2)}$ \hfill (6,Conversão-$\lambda$) \\
	8. \den{ele$_{1}$}$^{g}$ = g(1) \hfill (L) \\
	9. \den{DP$_{suj}$}$^{g}$ = \den{ele$_{1}$}$^{g}$ = \textit{g(1)}\hfill (8, NR) \\
	10. \den{S}$^{g}$ = \den{VP}$^{g}$(\den{NP$_{suj}$}$^{g}$) \hfill (AF) \\
	11. \den{S} = $(\lambda y.\ \predica{gosta}{y,g(2)})(\text{g(1)})$\hfill (7, 9, 10) \\
	12. \den{S}$^{g}$ = 1 \textit{sse} $\predica{gosta}{g(1),g(2)}$ \hfill (11,Conversão-$\lambda$)
\end{exe}

\n As condições de verdade de (\ref{las}) dependem, portanto, do
valor atribuído a \textit{g}. Por exemplo:

\begin{exe}
	\ex
	\begin{xlist}
		\ex \den{S}\gtwo{1}{Pedro}{2}{Maria} = 1 \textit{sse} Pedro gosta de Maria
		\ex \den{S}\gtwo{1}{João}{2}{Marta} = 1 \textit{sse} João gosta de Marta
		\ex \den{S}\gvaz = \textit{indefinido} 
	\end{xlist}
\end{exe}

\n No caso da atribuição $\varnothing$, o sistema não confere
valor de verdade à sentença. Isso deriva do fato de que as
extensões dos pronomes são indefinidas, o que faz com que todos os
constituintes que dominam esses pronomes, incluindo \textit{S},
tenham suas extensões indefinidas. Isso, por sua vez, segue do
fato de que nem o princípio dos nós não ramificados, nem aplicação
funcional podem ser utilizados para a obtenção das extensões
desses nós.

Terminamos assim nossa primeira incursão na semântica dos
pronomes. Voltaremos a falar deles e, sobretudo, do conceito de
atribuição nos próximos capítulos.

\bigskip

\begin{tcolorbox}[parbox=false,boxrule=0pt,sharp corners,breakable]

\section*{Sugestões de leitura}
\addcontentsline{toc}{section}{Sugestões de leitura}

\n  A semântica dos nomes próprios e das descrições definidas foi alvo da atenção de muitos filósofos, incluindo os notáveis e influentes Gottlob Frege, Bertrand Russell, P.F. Strawson, John Searle, Keith Donnellan, David Kaplan e Saul Kripke. Coletâneas reunindo os textos desses (e outros) autores sobre esses (e outros) temas ligados à filosofia da linguagem incluem \cite{ostertag98}, \cite{martinich01} e \cite{ludlow98}. Obras mais recentes sobre o tema incluem \cite{abbott10}, \cite{elbourne13} e \cite{neale90}. Para uma apresentação acessível e abrangente sobre a semântica dos pronomes, ver \cite{buring05}. Para um excelente panorama sobre a interpretação das descrições definidas (e indefinidas), ver \cite{heim91}.

\end{tcolorbox}

\bigskip

\begin{tcolorbox}[parbox=false,boxrule=0pt,sharp corners,breakable]

\section*{Exercícios}
\addcontentsline{toc}{section}{Exercícios}

\n\textbf{I.} Calcular, passo a passo, as condições de verdade da
sentença abaixo:\\

(1) A mãe do João adora ele$_{3}$.\\

\n\textbf{II.} Considere a seguinte sentença:\\

(1) Ele$_{1}$ é o João.\\

\n \textbf{(A)} Com base nas definições abaixo e assumindo que a
descrição definida \textit{o João} é um designador rígido, mostre
que não existe nenhuma atribuição \textit{g} tal que (1) seja
uma contingência em relação a \textit{g}:\\

\n (D1) \textit{S} é uma \textit{tautologia} em relação a
\textit{g} se, e somente se, para toda situação possível,
\den{S}$^{g}$ = 1.\\

\n (D2) \textit{S} é uma \textit{contradição} em relação a
\textit{g} se, e somente se, para toda situação possível,
\den{S}$^{g}$ = 0.\\

\n (D3) \textit{S} é uma \textit{contingência} em relação a
\textit{g} se, e somente se, existir pelo menos uma situação
possível tal que \den{S}$^{g}$ = 0 e pelo menos uma situação
possível tal que \den{S}$^{g}$ = 1.\\



\n \textbf{(B)} Mostre que se trocarmos as definições (D1)-(D3)
acima por (D1$'$)-(D3$'$) abaixo, a sentença (1) não será nem uma
tautologia, nem uma contradição, e nem uma contingência.\\

\n (D1$'$) \textit{S} é uma \textit{tautologia} se, e somente se,
para toda atribuição \textit{g} e para toda situação possível, \den{S}$^{g}$ = 1.\\

\vspace*{-1mm}
\n (D2$'$) \textit{S} é uma \textit{contradição} se, e somente se,
para toda atribuição \textit{g} e para toda situação possível,
\den{S}$^{g}$ = 0.\\

\vspace*{-1mm}
\n (D3$'$) \textit{S} é uma \textit{contingência} se, e somente
se, para toda atribuição \textit{g}, existir pelo menos uma
situação possível tal que \den{S}$^{g}$ = 0 e pelo menos uma
situação possível tal que \den{S}$^{g}$ = 1.\\


\n \textbf{(C)} Mostre que, se mantivermos (D1$'$) e (D2$'$) acima,
mas trocarmos (D3$'$) por (D3$''$) abaixo, a sentença (1) será uma
contingência:\\
\vspace*{-1mm}
\n (D3$''$) \textit{S} é uma \textit{contingência} se, e somente
se, existir pelo menos uma atribuição \textit{g} e uma situação
possível tal que \den{S}$^{g}$ = 0 e pelo menos uma atribuição
$g\prime$ e uma situação possível tal que \den{S}$^{g'}$ = 1.\\

\largerpage[2]
\n \textbf{(D)} Para cada um dos três conjuntos de definições
apresentados acima, diga se as sentenças abaixo são tautologias,
contradições ou contingências, justificando sempre sua resposta:\vspace*{-2mm}
\pagebreak 

(2) João nasceu no Brasil.\\

(3) Ele$_{1}$ nasceu no Brasil.\\

(4) João nasceu no Brasil e João não nasceu no Brasil.\\

(5) Ele$_{1}$ nasceu no Brasil e ele$_{1}$ não nasceu no Brasil.\\

(6) João nasceu no Brasil ou João não nasceu no Brasil.\\

(7) Ele$_{1}$ nasceu no Brasil ou ele$_{1}$ não nasceu no Brasil.\\

(8) João nasceu no Brasil e ele$_{1}$ não nasceu no Brasil.\\

(9) Ele$_{1}$ nasceu no Brasil e ele$_{2}$ não nasceu no Brasil.\\


\end{tcolorbox}




%%%%

%versao de 15-MAR-2019

\chapter{De volta aos pronomes}

Neste último capítulo, retornaremos à interpretação dos pronomes, dedicando-nos especialmente à \textsc{ligação pronominal} e sua relação com a \textsc{correferência} e o movimento sintático. Discutiremos também a possibilidade de uma \textsc{semântica sem variáveis}, em que a interpretação dos pronomes e a implementação da ligação pronominal se dão sem o uso de atribuições nem índices. 

\section{Pronomes livres e ligados}

No capítulo 4, introduzimos a noção de atribuição e discutimos seu papel na codificação de informação contextual relacionada a referentes tornados salientes no contexto. A partir disso, assumimos que as extensões dos pronomes pessoais variam de contexto a contexto e implementamos essa ideia através da atribuição de índices a esses pronomes e da relativização das extensões a uma atribuição. Essa relativização era herdada por constituintes maiores que dominavam os pronomes, podendo chegar à oração principal, como ilustrado abaixo:

\begin{exe}
	\ex Ela$_{1}$ ama Pedro.
\end{exe}

%\vspace{12pt}

\n\den{ama}$^{g}$ = $\lambda x.\lambda y.\ \predica{ama}{y,x}$

\n\den{Pedro}$^{g}$ = $pedro$

\n\den{ama Pedro}$^{g}$ = \den{ama}$^{g}$(\den{Pedro}$^{g}$)

\n\den{ama Pedro}$^{g}$ = $\lambda y.\ \predica{ama}{y,pedro}$

\n\den{ela$_{1}$}$^{g}$ = $g(1)$

\n\den{ela$_{1}$ ama Pedro}$^{g}$ = \den{ama Pedro}$^{g}$(\den{ela$_{1}$}$^{g}$)

\n\den{ela$_{1}$ ama Pedro}$^{g}$ = 1 \textit{sse} $\predica{ama}{g(1),pedro}$ \\

\n Nesse caso, o pronome $ela_{1}$ está livre e a sentença será pareada com condições de verdade dependentes da atribuição. Especificamente, apenas no caso de atribuições que tiverem o índice 1 em seu domínio, conferiremos um valor de verdade à sentença.

Já no capítulo 5, discutimos a interpretação composicional das orações relativas. Analisamos casos em que um pronome (resumptivo) era ligado por um índice numérico associado a um pronome relativo. Nesse caso, a interpretação da oração relativa como um todo não dependia mais de uma atribuição. Isso era consequência do princípio composicional que chamamos de abstração funcional, como ilustrado resumidamente abaixo:

\begin{exe}
\ex $[_{\text{NP}}\ \text{cachorro}\ [\text{\underline{que 1 Pedro cuida dele}}_{1}]]$\\
\den{1 Pedro cuida dele$_{1}$}$^{g}$ = $\lambda x.$ \den{Pedro cuida dele$_{1}$}$^{g[1 \rightarrow x]}$\\
\den{Pedro cuida dele$_{1}$}$^{g[1 \rightarrow x]}$ = 1 \textit{sse} \textsc{cuida}(\textit{pedro},$g[1 \rightarrow x](1)$)\\
\den{1 o Pedro cuida dele$_{1}$}$^{g}$ = $\lambda x.$ \textsc{cuida}(\textit{pedro},$x$)
\end{exe}

\n No restante desse capítulo, vamos olhar um pouco mais de perto para a ligação e sua relação com a correferência pronominal.  


\section{Correferência vs. ligação}

Consideremos o par de sentenças abaixo:

\begin{exe}
    \ex O João adora o chefe dele. \label{jo}
    \ex Só o João adora o chefe dele.  \label{sow}
\end{exe}

\n Fixemos nossa atenção em casos em que, intuitivamente, o pronome se refere a João, o mesmo referente do nome próprio que aparece na posição de sujeito da sentença. O fato a se notar é que, mesmo quando restringimos nossa atenção a esse tipo de interpretação, (\ref{sow}) é ambígua de uma forma que (\ref{jo}) não é. Imagine, por exemplo, que Pedro seja o chefe do João. Nesse caso, (\ref{jo}) é semanticamente equivalente a (\ref{jo1}) abaixo:
 
\begin{exe}
	\ex O João adora o Pedro. \label{jo1}
\end{exe}

\n Já (\ref{sow}) admite duas interpretações nesse contexto:

\begin{exe}
	\ex O João adora o Pedro e ninguém mais adora o Pedro. \label{so1}
	\ex O João adora o Pedro e ninguém mais adora o próprio
	chefe.  \label{so2}
\end{exe}

\n A primeira interpretação parece a mais saliente, mas basta trabalharmos um pouquinho o contexto para que a segunda emerja naturalmente, como, por exemplo, em (\ref{mel}):

\begin{exe}
	\ex Na firma em que o João trabalha, todos os funcionários se dão bem com os respectivos chefes. Mas, \textbf{só o João ADORA o chefe dele.} \label{mel}
\end{exe}


Vejamos como abordar esses fatos com o sistema interpretativo que temos à nossa disposição. Comecemos notando que já estaríamos muito próximos do que almejamos se conseguirmos atribuir ao predicado verbal \textit{adora o chefe dele} as duas interpretações em (\ref{duasa}) e (\ref{duasb}) abaixo, ambas funções de tipo $\langle e,t\rangle$:

\begin{exe}
\ex \den{adora o chefe dele} = \label{duas}
\begin{xlist}
\ex $\lambda x.\ \predica{adora}{x,o chefe do joão}$ \label{duasa}
\ex $\lambda x.\ \predica{adora}{x,o chefe de x}$ \label{duasb}
\end{xlist}
\end{exe}


\n Em (\ref{duasa}), dizemos que o pronome foi interpretado anaforicamente, sendo correferente ao sujeito João. Já em (\ref{duasb}), o pronome foi interpretado como uma \textit{variável ligada}. Note que, nessa representação, tanto a posição sintática correspondente ao pronome possessivo quanto a posição do sujeito do verbo \textit{adorar} estão preenchidas com uma mesma variável \textit{x}, ambas ligadas pelo mesmo operador lambda ($\lambda x$).

Se, agora, aplicarmos as funções em (\ref{duasa}) e (\ref{duasb}) à extensão do nome próprio \textit{João}, ou seja, ao indivíduo João, notaremos que ambas levarão ao mesmo resultado, como mostrado a seguir:

\begin{exe}
	\ex $(\lambda x.\ \predica{adora}{x,o chefe do joão})$(\den{João}) = 1 \textit{sse} $\predica{adora}{joão,o chefe do joão}$ \label{int1}
	\ex $(\lambda x.\ \predica{adora}{x,o chefe de x})$(\den{João}) = 1 \textit{sse} $\predica{adora}{joão,o chefe do joão}$  \label{int2}
\end{exe}

\n Dessa forma, quando o sujeito da oração em questão for um nome próprio, as interpretações anafórica e de variável ligada não levarão a diferenças de significado. Daí a não ambiguidade de (\ref{jo}) que apontamos mais acima. Já o caso de (\ref{sow}) é diferente, por não envolver um sujeito referencial. Para entender melhor o que está acontecendo, olhemos um pouco mais de perto para a interpretação do sintagma \textit{só o João} em uma sentença mais simples:

\begin{exe}
	\ex Só o João trabalha.  \label{simp}
\end{exe}

Intuitivamente, essa sentença é verdadeira se, e somente se, o João trabalha e ninguém mais trabalha. Ainda intuitivamente, podemos pensar na contribuição de \textit{só o João} para a interpretação de (\ref{simp}) como uma função que inspeciona a extensão de VP e verifica se João é o único indivíduo que é levado no valor 1 por esta função. Se esse for o caso, a sentença será verdadeira. Caso contrário, será falsa. Formalizando:

\begin{exe}
	\ex \den{só o João} = $\lambda F.\ F'(\textit{joão})\ \&\ \neg\exists x [x\neq \textit{joão}\ \&\ F'(x)]$ \\
	\den{só o João trabalha} = 1 \textit{sse}\\
	$\predica{trabalha}{joão}\ \&\ \neg\exists x [x\neq \textit{joão}\ \&\ \predica{trabalha}{x}]$	\label{soh}
\end{exe}

\n Não vamos aqui explicitar a extensão de \textit{só}. Como já mencionamos no capítulo 7, \textit{só} é sintaticamente bastante flexível, podendo combinar com uma grande variedade de constituintes. Em função disso e de sua interação com aspectos prosódicos que não estamos levando em conta neste livro, sua semântica não é trivial. De qualquer forma, para os propósitos desta exposição, cujo interesse maior é na ligação pronominal, o que temos em (\ref{soh}) nos basta.

Apliquemos, então, a extensão em (\ref{soh}), de tipo $\langle et,t\rangle$, às duas funções (a) e (b) de tipo $\langle e,t\rangle$ em (\ref{duas}):

\begin{exe}
	\ex \den{só o João}$(\lambda x.\ \predica{adora}{x,o chefe do joão})$ = 1 \textit{sse} \\
	$\predica{adora}{joão,o chefe do joão}\  \&\ \neg\exists x [ x\neq
	\textit{joão}\ \&\ \predica{adora}{x,o chefe do joão}]$ \label{duas1}
	
	\ex \den{só o João}$(\lambda x.\ \predica{adora}{x,o chefe de x})$ = 1 \textit{sse}\\
	 $\predica{adora}{joão,o chefe do joão}\ \&\ \neg\exists x [ x\neq \textit{joão}\ \&\ \predica{adora}{x,o chefe de x}]$  \label{duas2}
\end{exe} 

Como se pode ver, essas são exatamente as duas interpretações que havíamos detectado inicialmente, e nossa explicação para a ambiguidade de (\ref{sow}) e seu contraste com a não ambiguidade de (\ref{jo}) está praticamente pronta. Resta-nos apenas explicitar as representações sintáticas do VP \textit{adora o chefe do João} que conduzem às interpretações em (\ref{duas}). 

Para a interpretação de correferência, nada relevante precisa ser acrescentado. Aplicação funcional, envolvendo as extensões de VP e do DP sujeito, basta:\\

\Tree [ \qroof{só o João}.DP \qroof{odeia o chefe dele$_{1}$}.VP ].S

\bigskip

\n Nesse caso, o pronome está livre e basta assumirmos que a atribuição correspondente ao contexto relevante é tal que $g(1)$ = \den{João}.

Já para a interpretação de variável ligada, precisamos acionar abstração funcional, o que, por sua vez, implica a presença de um índice derivado de movimento sintático. Aqui podemos nos valer tanto da hipótese do sujeito interno a VP, que se move abertamente para sua posição superficial, quanto da regra de alçamento de quantificador, em que DPs se movem cobertamente para uma posição periférica na sentença, ambas vistas no capítulo anterior:\\

\Tree [.S \qroof{só o João}.DP [.$\alpha$ 1 \qroof{$t_1$ odeia o chefe dele$_{1}$}.VP ] ]
	
\bigskip

\n Nesse caso, o pronome e o vestígio estão coindexados e ambos atrelados ao mesmo índice resultante do movimento sintático. Abstração funcional se aplica nesse nível, e o constituinte $\alpha$ acima terá justamente a extensão que desejamos, ou seja, a interpretação vista em (\ref{duasb}).

Note, por fim, que a interpretação de correferência é compatível com o movimento sintático do sujeito, bastando que o pronome e o vestígio não estejam coindexados:\\

\Tree [.S \qroof{só o João}.DP [.$\alpha$ 2 \qroof{$t_2$ odeia o chefe dele$_{1}$}.VP ] ]

\bigskip

\n Fica a cargo do leitor calcular as condições de verdade dessa estrutura e conferir que são as mesmas obtidas na interpretação da estrutura sem movimento vista mais acima.

\section{Ligação sem índices nem atribuições}

Dado tudo o que vimos até aqui, pode parecer que, para lidar com a semântica dos pronomes em geral (e dos pronomes ligados em particular), índices, atribuições e uma regra nos moldes de abstração funcional sejam essenciais para captar a interpretação desses elementos. Surpreendentemente, isso não é verdade e, conforme mostraremos nesta seção, é possível formalizar a interpretação pronominal sem esse aparato. O que veremos a seguir é uma adaptação baseada em trabalhos de Pauline Jacobson, denominado \textsc{semântica livre de variáveis} (\textit{variable-free semantics}, em inglês). 
	
Para apresentarmos essa proposta semântica, voltaremos a uma das sentenças que discutimos nas seções anteriores:

\begin{exe}
	\ex Só o João odeia o chefe dele.  \label{deno}
\end{exe} 

\n Sua estrutura sintática é a seguinte:\\

\Tree [.S \qroof{só o João}.DP [.VP odeia [.DP o
	[.NP chefe [.PP de ele ] ] ] ] ]
	
\bigskip
	
\n Note, desde já, a ausência de índices e vestígios de movimento sintático. Comecemos pela interpretação dos pronomes (vamos continuar ignorando traços de gênero, número e pessoa). Nesse novo sistema, um pronome denota uma \textsc{função identidade}, que mapeia cada elemento de seu domínio no próprio elemento. No caso dos pronomes pessoais singulares que estamos discutindo, trata-se de uma função de tipo $\langle e,e\rangle$ que mapeia um indivíduo $x$ qualquer nesse próprio indivíduo $x$:

\begin{exe}
	\ex \den{ele} = $\lambda x_{e}.\ x$  \label{prono}
\end{exe}

\n Pronomes, portanto, não são mais expressões referenciais que denotam indivíduos. Não se preocupe se isso parece contraintuitivo. Devemos julgar uma ideia não pelas impressões iniciais, mas pelos seus resultados. Prossigamos, então, para ver aonde chegaremos. Assumindo que a preposição \textit{de} em (\ref{deno}) seja semanticamente vácua, o PP \textit{dele} herdará essa extensão:\\

\n \den{PP} = \den{ele} = $\lambda x.\ x$ \\

\n Vejamos, agora, como essa extensão se combina com a extensão do substantivo comum \textit{chefe}, do qual é complemento:\\

\n \den{chefe} = $\lambda x.\ \lambda y.\ \predica{chefe}{y,x}$ \\

\n Essa é uma função de tipo $\langle e,et\rangle$. Como a extensão do PP é de tipo $\langle e,e\rangle$, não podemos usar aplicação funcional. Em seu lugar, vamos usar um novo princípio, que chamaremos de \textsc{composição funcional}. Antes de defini-lo formalmente e aplicá-lo a nosso exemplo, vamos entender seu funcionamento informalmente por meio de um exemplo envolvendo números.

Considere, a título de ilustração, duas funções $F$ e $G$ que mapeiam números naturais em números naturais, definidas da seguinte forma:\\

\n $F:\ \lambda n.\ n^{2}$ \\
   $G:\ \lambda n.\ 3n$ \\
   
\n A função $F$ mapeia um número no seu quadrado, enquanto $G$ mapeia um número no seu triplo. Pense, agora, em uma outra função $H$, definida a partir de $F$ e $G$ da seguinte forma:\\

\n $H:\ \lambda n.\ G(F(n))$ \\

\n Trata-se também de uma função de números em números. Veja que o input de $H$ passa primeiro pela função $F$ e que o output de $F$ serve de input para $G$. O resultado final $G(F(x))$ é o output da função $H$. Como exemplo, tomemos $n=2$. Teremos então o seguinte:\\

\n $H(2) = G(F(2)) = G(2^{2}) = G(4) = 3 \cdot 4 = 12$ \\

\n Diz-se de uma função como $H$ que ela é a composição de $F$ e $G$. Note que, para que a composição funcione, é preciso que o output de uma das funções originais --- $F$, no caso --- pertença ao domínio da outra função ($G$). No caso acima, isso não foi problema pois ambas as funções eram funções de números para números. Note, por fim, que a ordem de composição é importante. No nosso exemplo, $G(F(x))$ e $F(G(x))$ são diferentes, como ilustrado abaixo:\\

\n $F(G(2)) = F(3 \cdot 2) = F(6) = 6^{2} = 36$ \\

Com essas preliminares no lugar, voltemos ao nosso exemplo (\ref{deno}) e sua derivação semântica. No ponto em que paramos, precisávamos combinar a extensão de N (\textit{chefe}), que é de tipo $\langle e,et\rangle$ com a extensão de PP (\textit{dele}), que é de tipo $\langle e,e\rangle$. Já sabemos que aplicação funcional não nos serve aqui. Mas note que composição funcional pode ser usada nesse caso para obtermos a extensão de NP (\textit{chefe dele}). O output de \den{PP} é de tipo $e$, justamente o tipo do input de \den{N}. Vejamos o resultado: \\

\n \den{NP} = $\lambda x_{e}.$ \den{N}(\den{PP}($x$))\\

\n Efetuando-se as devidas substituições e conversões lambda, teremos o seguinte:\\

\n \den{NP} = $\lambda x.\ \lambda y.\ \predica{chefe}{y,x}$ \\

\n Note que essa é a mesma extensão de N! De fato, compor com a função identidade é semanticamente vácuo. Se você estiver um pouco confuso, consulte nosso exemplo anterior envolvendo funções numéricas e assuma que $F$ seja a função identidade. Como tal, $F(n) = n$ e $G(F(n)) = G(n)$. Sendo $H$ a composição de $F$ e $G$, $H(n)=G(n)$ para qualquer $n$. Em outras palavras, $H$ e $G$ são idênticas.

Já podemos agora formalizar nosso novo princípio composicional (que ainda sofrerá uma pequena modificação):\\

\begin{tcolorbox}[boxrule=0pt,sharp corners]

\noindent \textbf{Composição funcional (versão preliminar)}

\n Seja $\alpha$ um nó ramificado, cujos constituintes imediatos
são $\beta$ e $\gamma$. Sejam $a$, $b$, $c$ três tipos  quaisquer. Se \den{$\beta$} é uma função de
tipo $\langle a,b \rangle$ e \den{$\gamma$} é uma função de
tipo $\langle b,c \rangle$, então
\den{$\alpha$} é uma função de tipo $\langle a,c \rangle$ e \den{$\alpha$} = $\lambda w_{a}.$\den{$\gamma$}(\den{$\beta$}($w$)).

\end{tcolorbox}

\bigskip


\n O real papel da composição funcional na interpretação pronominal fica mais claro quando lidamos com constituintes que contêm pronomes, como o NP \textit{chefe dele}, que é justamente o ponto em que estamos. Esse NP, cuja extensão vista acima é de tipo $\langle e,et\rangle$ precisa combinar com o artigo definido \textit{o}, cuja extensão é de tipo $\langle et,e\rangle$, para obtermos a extensão do DP resultante. Novamente, aplicação funcional não pode ser usada, mas composição funcional sim. Vejamos:\\

\n \den{DP} = $\lambda x_{e}.$ \den{o}(\den{chefe dele}(\textit{x}))

\n \den{o} = $\lambda F.\ \iota y[F'(y)]$ 

\n \den{chefe dele}($x$) = $\lambda y.\ \predica{chefe}{y,x}$

\n \den{DP} = $\lambda x_{e}.\ \iota y[\predica{chefe}{y,x}]$\\

\n Ou, de maneira mais direta:\\

\n \den{DP} = $\lambda x.\ \textit{o chefe de x}$\\

\n Olhando para esse resultado, fica claro que composição funcional está ``passando adiante'' a posição argumental ocupada pelo pronome. Note que a extensão do DP \textit{o chefe dele} é de tipo $\langle e,e\rangle$ e que se aplicássemos essa função a um indivíduo $i$ qualquer, o resultado seria o chefe desse indivíduo $i$. 

No próximo passo interpretativo, a extensão do DP objeto irá combinar com a extensão do verbo transitivo \textit{adora}, que representamos abaixo:\\

\n \den{adora} = $\lambda x.\lambda y.\ \predica{adora}{y,x}$ \\

\n É nesse momento que a ligação do pronome irá ocorrer semanticamente. O resultado que buscamos para o VP resultante é o seguinte:\\

\n \den{adora o chefe dele} = $\lambda x.\ \predica{adora}{x,o chefe de x}$ \\

\n Note por essa representação que a posição de sujeito (argumento externo) do verbo e a posição argumental original do pronome estão preenchidas pela mesma variável $x$ e ligadas pelo mesmo operador lambda. É isso, efetivamente, que corresponde ao processo de ligação do ponto de vista semântico. No sistema dos capítulos e seções anteriores, isso era feito via abstração funcional com o apoio de índices e atribuições. Na ausência de tudo isso, um novo princípio precisa ser formulado que tenha o efeito de identificar essas posições argumentais que acabamos de mencionar. E aqui fica mais claro o papel que a composição funcional desempenhou. Ao passar adiante o argumento correspondente à posição original do pronome, ela permitiu que o mesmo chegasse até o constituinte (DP) que é irmão do verbo na estrutura sintática: \\

\n \Tree [.VP [.$\lambda x.\lambda y.\ y\ \text{adora}\ x$ adora ] [.$\underline{\lambda x}.\ \iota y[\predica{chefe}{y,\underline{x}}]$ [.$\lambda F.\ \iota y[F'(y)]$ o ] [.$\underline{\lambda x}.\lambda y.\ \predica{chefe}{y,\underline{x}}$ [.$\lambda x.\lambda y.\ \predica{chefe}{y,x}$ chefe ] [.$\underline{\lambda x}.\ \underline{x}$ dele ] ] ] ]

\bigskip


\n Como a extensão do verbo carrega a outra posição argumental em questão, podemos tirar partido disso e manipular ambas na formulação do princípio que chamaremos de \textsc{ligação funcional}, definido preliminarmente abaixo:\\

\begin{tcolorbox}[boxrule=0pt,sharp corners]

\noindent \textbf{Ligação funcional (versão preliminar)}

\n Seja $\alpha$ um nó ramificado, cujos constituintes imediatos
são $\beta$ e $\gamma$. Seja $a$ um tipo qualquer. Se \den{$\beta$} é uma função de
tipo $\langle a,et \rangle$ e \den{$\gamma$} é uma função de
tipo $\langle e,a \rangle$, então
\den{$\alpha$} é uma função de tipo $\langle e,t \rangle$ e \den{$\alpha$} = $\lambda x_{e}.\ $\den{$\beta$}(\den{$\gamma$}($x$))($x$).

\end{tcolorbox}

\bigskip

\n Apliquemos esse princípio composicional na computação da extensão do VP \textit{adora o chefe dele}. Nesse caso, $a$ acima é o tipo $e$ e a extensão do verbo \textit{adora} -- de tipo $\langle e,et \rangle$ -- fará o papel de $\beta$. Já a extensão de \textit{o chefe dele} -- de tipo $\langle e,e \rangle$ -- fará o papel de $\gamma$:\\

\n \den{adora o chefe dele} = $\lambda x.\ $\den{adora}(\den{o chefe dele}($x$))($x$).\\

\n Olhando para o que aparece nessa expressão após o prefixo $\lambda x.$, percebe-se que a extensão do verbo será saturada pelos indivíduos \textit{o chefe de x} e \textit{x}. O resultado é exatamente o que queríamos:\\

\n \Tree [.$\lambda x.\ \predica{adora}{x,o chefe de x}$ [.V$_{\langle e,et\rangle}$ $(\lambda x.\lambda y.\ \predica{adora}{y,x})$ ] [.DP$_{\langle e,e\rangle}$ $(\lambda x.\ \text{\textit{o chefe de}}\ x)$ ] ]

\bigskip

\n Como se pode notar, trata-se de uma extensão de tipo $\langle e,t\rangle$. Como passo final, para derivarmos as condições de verdade da sentença, basta combinarmos essa extensão do VP \textit{adora o chefe dele} com a extensão do DP sujeito \textit{só o João}, utilizando aplicação funcional, exatamente como já fizemos na seção anterior. O resultado é a leitura de variável ligada de acordo com a qual João, e ninguém mais, tem a propriedade de adorar o próprio chefe.

\subsection{Limitando composição e ligação funcional}

Conforme acabamos de ver, o trabalho conjunto de ligação funcional e composição funcional (além, claro, de aplicação funcional) permitiu derivar a leitura de variável ligada para um pronome sem o uso de atribuições nem índices. Entretanto, na forma como está, o sistema resultante gera interpretações que não são atestadas, inclusive em casos que não envolvem pronomes. Considere, por exemplo, a sentença abaixo:

\begin{exe}
	\ex João odeia todo mundo. \label{err}
\end{exe}

Trata-se de uma oração simples com um verbo transitivo, um nome próprio na posição de sujeito e um DP quantificador na posição de objeto. Já vimos no capítulo anterior como interpretá-la, seja via alçamento de quantificador, seja via mudança de tipos. Mas, agora, com a composição funcional no nosso leque de regras composicionais, uma derivação alternativa emerge. Acompanhe, passo a passo, a derivação a seguir:\\

\n \den{odeia} = $\lambda x. \lambda y.\ \predica{odeia}{y,x}$ \hfill tipo $\langle e,et \rangle$ 

\n \den{todo mundo} = $\lambda F.\ \forall z [F'(z)]$ \hfill tipo $\langle et,t \rangle$

\n \den{odeia todo mundo} = $\lambda x.$ \den{todo mundo}(\den{odeia}(\textit{x})) \hfill (CF)

\n \den{odeia todo mundo} = $\lambda x.\ \forall z [z\ \predica{odeia}{z,x}]$ 

\n \den{João odeia todo mundo} = \den{odeia todo mundo}(\den{João}) \hfill (AF)

\n \den{João odeia todo mundo} = 1 \textit{sse} $\forall z [\predica{odeia}{z,joão}]$ \\

\n Inverteram-se os papeis argumentais! O sujeito terminou sendo interpretado como argumento interno e o objeto direto quantificado vinculado ao argumento externo do verbo.  

Note que, no exemplo anterior a esse, nosso objetivo com a composição funcional foi o de passar adiante a posição argumental de um pronome até que a posição do ligador aparecesse e a ligação funcional se aplicasse. O que precisamos agora é limitar seu uso a casos envolvendo constituintes que contêm pronomes. Fazer isso em um sistema guiado exclusivamente por tipos semânticos (\textit{type-driven}) como o nosso não é trivial e a solução a seguir é apenas um paliativo. 

Salientemos, de partida, que nesse nosso novo sistema com composição e ligação funcional, pronomes e constituintes que os contêm denotam funções que levam indivíduos (tipo \textit{e}) em algum tipo de entidade (tipo $a$). Para identificar esses constituintes, não associaremos tais denotações ao tipo $\langle e,a \rangle$ como fizemos até aqui, mas sim a um novo tipo  $\langle e;a \rangle$. Note a troca da vírgula por um ponto e vírgula. Acabamos de criar um novo construtor de tipos semânticos complexos (funcionais). Pronomes, por exemplo, serão de tipo $\langle e;e\rangle$. Podemos, assim, redefinir a composição funcional:\\

\begin{tcolorbox}[boxrule=0pt,sharp corners]

\noindent \textbf{Composição funcional}

\n Seja $\alpha$ um nó ramificado, cujos constituintes imediatos
são $\beta$ e $\gamma$. Sejam $a$ e $b$ dois tipos quaisquer. Se \den{$\beta$} é uma função de
tipo $\langle e;a \rangle$ e \den{$\gamma$} é uma função de
tipo $\langle a,b \rangle$, então
\den{$\alpha$} é uma função de tipo $\langle e;b \rangle$ e \den{$\alpha$} = $\lambda x_{e}.$\den{$\gamma$}(\den{$\beta$}($x$)).

\end{tcolorbox}

\bigskip

\n Composição funcional passa, agora, a ser utilizável só em casos que envolvem constituintes que contenham pronomes. De imediato, concluímos que ela não será acionada no caso de sentenças como (\ref{err}) acima, que não contêm pronomes. 

Também a ligação funcional pode ser limitada a casos com pronomes:\\

\begin{tcolorbox}[boxrule=0pt,sharp corners]

\noindent \textbf{Ligação funcional}

\n Seja $\alpha$ um nó ramificado, cujos constituintes imediatos
são $\beta$ e $\gamma$. Seja $a$ um tipo qualquer. Se \den{$\beta$} é uma função de
tipo $\langle a,et \rangle$ e \den{$\gamma$} é uma função de
tipo $\langle e;a \rangle$, então
\den{$\alpha$} é uma função de tipo $\langle e,t \rangle$ e \den{$\alpha$} = $\lambda x_{e}.\ $\den{$\beta$}(\den{$\gamma$}($x$))($x$).

\end{tcolorbox}

\bigskip

\n Note que, após a ligação, a extensão do constituinte resultante é de tipo $\langle e,t \rangle$ e não de tipo $\langle e;t \rangle$, indicando que o pronome em questão não está mais disponível para ser ligado ou passado adiante.

		
\subsection{Pronomes livres novamente}

Em nosso sistema anterior, com atribuições e índices, a extensão de sentenças com pronomes livres, como (\ref{free}) abaixo, correspondia a um valor de verdade e era relativizada a uma atribuição.

\begin{exe}
	\ex João gosta dela. \label{free}
\end{exe}

\n Agora, que não dispomos mais de atribuições, algo diferente acontece. Vejamos passo a passo:\\

\n \den{gosta} = $\lambda x. \lambda y.\ \predica{gosta}{y,x}$ \hfill tipo $\langle e,et \rangle$

\n \den{dela} = $\lambda x.\ x$ \hfill tipo $\langle e;e \rangle$

\n \den{gosta dela} = $\lambda x.$ \den{gosta}(\den{dela}(\textit{x})) \hfill (CF)

\n \den{gosta dela} = $\lambda x. \lambda y.\ \predica{gosta}{y,x}$ \hfill tipo $\langle e;et \rangle$\\

\n Compare os tipos das denotações do verbo \textit{gosta} e do VP \textit{gosta dela}. São diferentes, já que apenas o segundo contém um pronome. Note, agora, que, como o tipo do VP é $\langle e;et \rangle$, e não $\langle e,et \rangle$, aplicação funcional não pode ser usada para combinar VP com o sujeito \textit{João}, de tipo \textit{e}. Nesse caso, nem mesmo composição funcional pode ser usada, já que um dos constituintes não denota uma função! Mas há uma solução. Lembre-se de que, no capítulo anterior, vimos que nomes próprios podem ser tratados como quantificadores generalizados (tipo $\langle et,t \rangle$). Para o sujeito de nossa sentença, teríamos:\\

\n \den{João} = $\lambda F_{\langle e,t\rangle}.\ F'(\textit{joão})$\\

\n Sendo essa uma extensão de tipo $\langle et,t \rangle$, ela pode se combinar com a extensão do VP \textit{gosta dela}, de tipo $\langle e;et \rangle$, via composição funcional:\\

\n \den{João gosta dela} = $\lambda x.$ \den{João}(\den{gosta dela}(x)) \hfill (CF)\\

\n \den{João gosta dela} = $\lambda x.\ \predica{gosta}{joão,x}$  \hfill tipo $\langle e;t \rangle$\\ 

\n Note que a sentença não tem por extensão um valor de verdade, mas sim uma função de indivíduos em valores de verdade. A ideia é que, quando uma tal sentença for enunciada em um contexto em que um indivíduo esteja saliente, obteremos um valor de verdade aplicando a função a esse indivíduo.

Terminamos, assim, nossa breve exposição simplificada de uma semântica que lida com referência e ligação pronominal sem atribuições nem índices. Note que limitamos nossa atenção a casos com apenas um pronome ligado ou livre e que não discutimos casos com mais de um pronome, sejam eles todos ligados, todos livres, ou alguns livres e outros ligados. Para tanto, seria necessário generalizar os princípios de composição e ligação funcional que apresentamos acima. Não se trata de algo conceitualmente complexo, mas que deixaremos a cargo do leitor interessado verificar através das sugestões de leitura que apresentamos a seguir. 

Note, por fim, que, por um lado, esse novo sistema dispensa o uso de atribuições e índices, além da regra de abstração funcional. Por outro lado, necessita de duas novas regras,  composição e ligação funcional. Podemos declarar um vencedor conceitual? Difícil dizer.

% % % % % % %

\bigskip

\begin{tcolorbox}[parbox=false,boxrule=0pt,sharp corners,breakable]

\section*{Sugestões de leitura}
\addcontentsline{toc}{section}{Sugestões de leitura}

Discussões acessíveis sobre referência e ligação pronominal podem ser encontradas em \cite{heikra98} capítulos 9-11 e \cite{buring05}. A melhor introdução à semântica sem variáveis é \cite{jacobson14}, capítulo 15. O leitor interessado deve olhar os vários artigos de Pauline Jacobson (a principal defensora dessa abordagem) relacionados ao tema. Recomenda-se, em particular, \cite{jacobson99}.

\end{tcolorbox}

\bigskip

\begin{tcolorbox}[parbox=false,boxrule=0pt,sharp corners,breakable]

\section*{Exercícios}
\addcontentsline{toc}{section}{Exercícios}

\n \textbf{I.} Considere a sentença abaixo:\\

\n (1) \ \ \ Toda criança sente saudade da mãe dela.\\

\n Essa sentença admite uma leitura em que o pronome é interpretado como uma variável ligada. Derive passo a passo essa leitura tanto no sistema com índices e atribuições quanto no sistema livre de variáveis. [Para efeitos deste exercício, considere \textit{sente saudade de} como sendo um único item lexical, semelhante a um verbo transitivo] \\

\n \textbf{II.} Tendo o exercício anterior em mente, considere agora a seguinte sentença:\\

\n (2) \ \ \ A mãe dela sente saudade de toda criança.\\

\n Diferentemente de (1), (2) não admite para o pronome \textit{ela} uma leitura de variável ligada. A questão é se os sistemas que vimos nesse capítulo geram ou não tal leitura. Responda essa questão para os sistemas com e sem índices e atribuições. Para tanto, você precisará revisar o que vimos no capítulo anterior sobre DPs quantificadores na posição de objeto e as soluções baseadas em movimento e mudança de tipos.\\

\n \textbf{III.} Considere as duas sentenças abaixo:\\

\n (3) \ \ \ Pedro se viu (no espelho).\\
(4) \ \ \ Pedro viu ele (no espelho). \\

\n No exercício IV do capítulo 2 foi proposta uma entrada lexical para o pronome \textit{se} que captava seu caráter inerentemente reflexivo. Após rever esse exercício, mostre que a mesma leitura reflexiva pode ser atribuída a (4) pelo sistema sem variáveis que vimos neste capítulo. Por fim, diga se, em seu dialeto, (4) admite, ao lado de (3), essa interpretação.

\end{tcolorbox} 

 

% % % %

\addchap{Prefácio}
\begin{refsection}

%versao de 15-MAR-2019

%content goes here
\n O sistema interpretativo que desenvolveremos neste livro parte de três considerações, até certo ponto, intuitivas. A primeira delas é a de que saber o significado de uma sentença (declarativa) parece relacionado a saber as condições para que a sentença seja verdadeira. Se sabemos o significado da sentença \textit{tem mais de dez pessoas dentro desta sala}, podemos não saber se ela é verdadeira ou falsa, mas sabemos como o mundo deve ser para que ela seja verdadeira. Sabemos, por exemplo, que é necessário que o número de pessoas no interior da sala seja superior a dez, mas que o numero exato não é relevante. Sabemos, por outro lado, que é irrelevante haver ou não pessoas próximas à sala, ou as pessoas dentro dela serem homens ou mulheres, ou a sala ser grande ou pequena. Enfim, saber o significado de uma sentença se assemelha a saber as condições necessárias e suficientes para que a sentença seja verdadeira, ou, posto de maneira mais concisa, saber suas condições de verdade. Tendo isso em mente, exploraremos a ideia de que as condições de verdade de uma sentença sejam o seu próprio significado. 

A segunda consideração é que o significado de uma sentença é produto do significado das palavras que a compõem e da maneira como estas palavras estão agrupadas, ou seja, da estrutura sintática da sentença. Assim, não é surpresa que o significado de \textit{Pedro beijou Maria} difere do significado tanto de \textit{Pedro abraçou Maria} quanto de \textit{Maria beijou Pedro}. No primeiro caso, são palavras distintas e, no segundo, são as mesmas palavras arranjadas de maneira distinta. Radicalizando um tanto esta intuição, parece interessante explorar a ideia de que o significado de uma sentença dependa exclusivamente das palavras e da estrutura sintática que compõem a sentença. 

A terceira e última consideração é que expressões linguísticas como palavras, sintagmas nominais e verbais, além das próprias sentenças nos remetem a certas entidades ou a certos aspectos do mundo. Uma possibilidade bastante razoável para começar a compreender esse fato é assumir que as expressões linguísticas se relacionam, elas mesmas, direta ou indiretamente, com as coisas ou fatos do mundo, ou seja, elas têm referência. A que exatamente elas referem, é algo a ser decidido caso a caso. Assim, buscaremos um sistema que seja referencial ou, num jargão mais técnico, extensional, mas sem com isso igualarmos significado e referência (extensão). Ou seja, queremos capturar a intuição de que os sintagmas nominais  \textit{o presidente do Brasil em 2006} e \textit{o sucessor do presidente FHC} se referem ao mesmo indivíduo, Lula, mas que nem por isso têm o mesmo significado e, portanto, contribuem de maneiras distintas na derivação das condições de verdade das sentenças em que ocorrem.

Em suma, desenvolveremos um sistema em que o significado de uma sentença equivalha a suas condições de verdade, que seja composicional e extensional. No primeiro capítulo, desenvolveremos um pouco mais essas considerações iniciais. Nos demais, apresentaremos um sistema que atenda a essas considerações. Faremos isso, explicitando como ele é capaz de interpretar uma série de construções e sentenças do português. Antes, porém, algumas breves palavras sobre a concepção e o público alvo deste curso de semântica formal.

Este livro originou-se de uma apostila preparada para um curso de uma semana de introdução à semântica formal, oferecido na Escola de Verão em Linguística Formal (EVELIN), que aconteceu na Universidade Estadual de Campinas em janeiro de 2004. A partir daí, a apostila foi se expandindo e tornou-se a base de um curso semestral de introdução à semântica formal que tem sido oferecido regularmente no programa de pós-graduação do Departamento de Linguística da Universidade de São Paulo. Partes do conteúdo também foram utilizadas em mini-cursos oferecidos na Universidade Federal de Belo Horizonte em 2008, na Universidade Federal do Rio Grande do Sul em 2009, na Universidade Federal do Paraná, como parte do instituto organizado pela Associação Brasileira de Linguística em 2010, e novamente na Universidade Estadual de Campinas, durante a Escola de Inverno em Linguística Formal, em 2013. 

As reações (positivas e negativas) dos alunos em todas essas ocasiões foram cruciais no aprimoramento do material e sou extremamente grato a todos eles por essa enorme contribuição. Suas dificuldades, dúvidas e sugestões motivaram inúmeras revisões de forma e conteúdo, que, espero, tenham tornado o texto mais claro e fácil de acompanhar.

Como dito mais acima, o livro tem o intuito de apresentar um sistema semântico composicional, formalizado através de algumas ferramentas ló\-gi\-co-ma\-te\-má\-ti\-cas. Não se pressupõe experiência prévia com abordagens formais para o significado. Entretanto, espera-se que o leitor já tenha passado por uma introdução geral à linguística e seus níveis de análise (fonologia, morfologia, sintaxe), bem como a uma discussão informal sobre alguns temas centrais sobre o significado (semântica e pragmática). Em sala de aula, poderá ser usado tanto em cursos mais avançados na graduação quanto em cursos de pós-graduação. Fora dela, poderá satisfazer estudantes autodidatas, professores e pesquisadores não apenas de linguística, mas também de áreas afins com interesse na análise e formalização do significado no âmbito das línguas naturais, como filosofia, ciências cognitivas e da computação.

O material empírico coberto nos seis capítulos que seguem o capítulo introdutório (capítulo 1) compreende fenômenos relacionados a predicação (capítulo 2), coordenação e negação (capítulo 3), referência (capítulo 4), modificacão (capítulo 5), quantificação (capítulo 6) e ligação (capítulo 7). Eles devem ser lidos nessa ordem, já que em cada um deles, novas ferramentas analíticas serão apresentadas e, a partir daí, pressupostas e reutilizadas com frequência.   


O leitor que tiver assimilado o conteúdo aqui apresentado será capaz de aplicar as técnicas que aprendeu a outros domínios que não serão cobertos, como tempo, aspecto, e modalidade. Será também capaz de compreender boa parte da literatura mais técnica que se encontra nos principais periódicos da área. Ao final de cada capítulo, há sugestões de leitura para os que quiserem aprofundar-se nos temas abordados no texto. Há também exercícios que devem ser feitos logo após a leitura de cada capítulo e que são cruciais para a fixação do conteúdo em questão.\\




São Paulo, Março de 2019

%\printbibliography[heading=subbibliography]
\end{refsection}

